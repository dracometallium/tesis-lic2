% vim: set spell spelllang=es syntax=tex :

\section{Aplicación de la computación paralela en los sistemas de visión por computadora}

\label{algoritmosParalelosYVision}

En la actualidad la mayoría de los equipos de escritorio tienen más de un solo
núcleo de procesamiento. Según la encuesta de hardware de la plataforma de
distribución digital \emph{Steam} de marzo del 2017 \cite{steamSurvey}, sólo
el $1.93$\% de los equipos donde está instalado posee un solo núcleo. Esto se
ve acompañado de la capacidad actual de las placas aceleradoras de video de
realizar procesamiento de propósito general permitiendo el procesamiento
masivamente paralelo en una computadora personal de gama media. Esta tendencia
se extiende también a otros equipos de computación como computadoras
portátiles, celulares y consolas de video juegos, ya sean portátiles o no.

Reconociendo la importancia del desarrollo de aplicaciones paralelas, la Red
de Universidades Nacionales con Carreras Informáticas (\emph{RedUNCI}) incluye
en la lista de descriptores curriculares \emph{Algoritmos secuenciales,
concurrentes, distribuidos y paralelos}, \emph{Concurrencia y paralelismo}, y
\emph{Arquitecturas multiprocesador} \cite{RedUNCI2015}. El interés en estos
descriptores se ve reflejado en el hecho de que se recomiendan para las
carreras de grado Licenciatura en Ciencias de la Computación, Licenciatura en
Sistemas de Información, y Licenciatura en Informática. Las dos primeras son
dictadas actualmente en la facultad de informática de la \texttt{Universidad
Nacional del Comahue}, mientras que la tercera se encuentra en proceso de
diseño.

Dado el gran volumen de datos que deben manejar, los sistemas de visión por
computadora se ven muy beneficiados por las capacidades de paralelismo del
hardware actual. En 2004 se propuso por primera vez un algoritmo que implementa
una red neuronal artificial de dos capas sobre una placa aceleradora de video
\cite{GPUforMLA}. Desde ese entonces el uso de placas aceleradoras de video ha
demostrado su utilidad en la ejecución de redes neuronales de convolución. Este
tipo especifico de redes neuronales artificiales, cuyo funcionamiento está
basado en el córtex visual, son especialmente útiles para resolver problemas de
visión por computadora \cite{usingCCN4IR2015}. Su capacidad de reconocimiento y
su velocidad de respuesta cuando se ejecutan en sistemas masivamente paralelos
permiten ser utilizadas incluso para el control de vehículos autónomos
\cite{e2eLearning4SDC}.

Además del incremento de las capacidades de procesamiento, se ha visto un
incremento en la capacidad de captura de información: dispositivos como cámaras
y acelerómetros son cada vez más baratos y pequeños. Dado que los sensores
pueden estar montados sobre dispositivos con reducido poder de procesamiento,
normalmente los datos son enviados a ``la nube'', lo cual puede ser problemático
por los límites de ancho de banda y las dudas respecto de la privacidad de los
datos.

Para el caso particular donde el procesamiento se realiza a través de redes
neuronales de convolución, en \cite{pipelinebasedCaffe2017} se propuso un
esquema que permite atenuar ambos problemas. El esquema consiste en dividir la
red neuronal en dos partes; las primeras $N$ capas son ejecutadas en el
dispositivo donde están montados los sensores, y el resto de las capas son
ejecutadas en un servidor remoto. El éxito de este trabajo plantea la
posibilidad de dividir el trabajo en más capas, ejecutando capas intermedias en
una computadora del usuario dentro de la red local (o incluso un teléfono
celular, ya que en la actualidad cuentan normalmente con por lo menos cuatro
núcleos) con el fin de enviar al servidor externo menor cantidad de datos e
información sensible.

Sin dudas, la computación paralela es una enorme contribución para el éxito de
la visión por computadora, y los futuros profesionales deberán estar capacitados
para usar estas tecnologías.
