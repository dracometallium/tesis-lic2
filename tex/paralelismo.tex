% vim: set spell spelllang=es syntax=tex :

\section{Algoritmos paralelos y visión por computadora}

En la actualidad la mayoría de los equipos de escritorio tienen mas de un solo
núcleo de procesamiento. Según la encuesta de hardware de la plataforma de
distribución digital \emph{Steam} de marzo del 2017 \cite{steamSurvey}, solo
el $1.93$\% de los equipos donde esta instalado posee un solo núcleo. Esto se
ve acompañado de la capacidad actual de las placas aceleradoras de vídeo de
realizar procesamiento de propósito general permitiendo el procesamiento
masivamente paralelo en una computadora personal de gama media. Esta tendencia
se extiende también a otros equipos de computación como computadoras
portátiles, celulares y consolas de vídeo juegos, ya sean portátiles o no.

Reconociendo la importancia del desarrollo de aplicaciones paralelas, la Red
de Universidades Nacionales con Carreras Informáticas (\emph{RedUNCI}) incluye
en la lista de descriptores curriculares \emph{Algoritmos secuenciales,
concurrentes, distribuidos y paralelos}, \emph{Concurrencia y paralelismo}, y
\emph{Arquitecturas multiprocesador} \cite{RedUNCI2015}. El interés en estos
descriptores se ve reflejado en el hecho de que se recomiendan para las
carreras de grado Licenciatura en Ciencias de la Computación, Licenciatura en
Sistemas de Información, y Licenciatura en Informática. Las dos primeras son
dictadas actualmente en la facultad de informática de la \texttt{Universidad
Nacional del Comahue}, mientras que la tercera se encuentra en proceso de
diseño.

Dado el gran volumen de datos que deben manejar y su tendencia a la localidad
espacial, los sistemas de visión por computadora se ven muy beneficiados por
las capacidades de paralelismo del hardware actual. En 2004 se propuso por
primera vez un algoritmo implementando una red neuronal artificial de dos
capas ejecutando en una placa aceleradora de vídeo \cite{GPUforMLA}. Desde ese
entonces el uso de placas aceleraras de vídeo han demostrado su utilidad en la
ejecución de redes neuronales de convolución. Este tipo especifico de redes
neuronales artificiales, cuyo funcionamiento esta basado en el cortex visual,
son especialmente útiles para resolver problemas de visión por computadora. Su
capacidad de reconocimiento y su velocidad de respuesta cuando se ejecutan en
sistemas masivamente paralelos permiten ser utilizadas incluso para el control
de vehículos autónomos \cite{e2eLearning4SDC}.

Un sistema de visión por computadora paralelo permitiría introducir a los
estudiantes a los conceptos de visión por computadora con un enfoque
actualizado, permitiéndoles no solo experimentar con distintos algoritmos de
visión por computadora, sino que explorar diferentes técnicas de paralelismo.

