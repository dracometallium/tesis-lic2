% vim: set spell spelllang=es syntax=tex :

\section{Importancia del estudio del desarrollo de aplicaciones paralelas}

En la actualidad la mayoría de los equipos de escritorio tienen mas de un solo
núcleo de procesamiento. Según la encuesta de hardware de la plataforma de
distribución digital \emph{Steam} de marzo del 2017 \cite{steamSurvey}, solo
el $1.93$\% de los equipos donde esta instalado posee un solo núcleo. Esta
tendencia se extiende también a otros equipos de computación como portátiles,
celulares y consolas de vídeo juegos ya sean portátiles o no.

Reconociendo la importancia del desarrollo de aplicaciones paralelas, la red
de universidades nacionales con carreras informáticas (\emph{RedUNCI}) incluye
en la lista de descriptores curriculares \emph{Algoritmos secuenciales,
concurrentes, distribuidos y paralelos}, \emph{Concurrencia y paralelismo}, y
\emph{Arquitecturas multiprocesador} \cite{RedUNCI2015}. El interés en estos
descriptores se ve reflejado en el hecho de que se recomiendan para dos de las
carreras de grado dictadas actualmente en la facultad de informática de
\texttt{Universidad Nacional del Comahue}, Licenciatura en Ciencias de la
Computación y Licenciatura en Sistemas de Información, así como en la carrera
en proceso de diseño Licenciatura en Informática.
