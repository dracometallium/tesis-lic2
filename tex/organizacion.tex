% vim: set spell spelllang=es syntax=tex :

\section{Organización del trabajo}

Este trabajo de tesis fue estructurado en seis capítulos. En el capítulo
\ref{marcoTeorico} se presenta el marco teórico necesario para este trabajo. La
sección \ref{mt_visionComputadora} es una introducción a la visión por
computadora y las etapas que suelen definirse en un sistema de visión por
computadora. En la sección \ref{descripcionSistemaBase} se describe el sistema de
visión global utilizado como base para el desarrollo del sistema propuesto en
este trabajo. En la sección \ref{mt_modelosparalelos} se presentan dos
clasificaciones que permiten distinguir los modelos computacionales de los
sistemas de cómputo paralelos. A continuación, en la sección \ref{mt_openmp} se
hace una introducción a \emph{OpenMP} y se explican sus modelos de computación y
las directivas utilizadas por el sistema propuesto.

En el capítulo \ref{sistemaPropuesto} se describe el sistema propuesto. La
sección \ref{descripcionSistema} especifican las tareas del sistema y la forma en
la cual se fragmentan los datos. Luego, en la sección
\ref{implementacionFramework}, se detalla la implementación.

En el capítulo \ref{experimentacion} se describen los experimentos realizados
utilizando el sistema desarrollado. Se comienza explicando la metodología
experimental en la sección \ref{metodologiaExperimental}, la plataforma
experimental se especifica en la sección \ref{plataformaExperimental}, y los
experimentos y sus resultados son discutidos en la sección \ref{resultados}.

En el capítulo \ref{usoEducativo} se plantean posibles usos del sistema como
herramienta didáctica para materias de sistemas paralelos, en la sección
\ref{eduparalelos}, y visión por computadora, en la sección \ref{eduvision}.

Finalmente en el capítulo \ref{conclucionesYTrabajosFuturos} se presentan las
conclusiones del trabajo, en la sección \ref{concluciones}, las publicaciones
derivadas en \ref{publicacionesDerivadas},  y los posibles trabajos futuros, en
la sección \ref{trabajosFuturos}.
