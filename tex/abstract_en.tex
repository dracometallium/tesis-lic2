% vim: set spell spelllang=en syntax=tex :
\ \\
\ \\
\label{pagsumm}
\noindent{\LARGE \sc Abstract}\\
\ \\
\ \\

\ \\

\ \\
\ \\

The \emph{RoboCup} (short for \emph{Robot World Cup}) is a tournament where two
teams play a simplified version of football. Its aim is to provide a platform
where the advances on multiple fields like artificial intelligence, computer
vision, and robotics can be put to a test. The \emph{RoboCup} has five leagues
ranging from simulated robots on a virtual field, up to humanoid robots with
local vision. Its most ancient league is the \emph{Small Size League}, also
called the \emph{SSL}.

On the \emph{SSL} games, the global vision system is shared by both teams. The
system process the video frames and reports the robots' and ball's position and
orientation. We propose an alternative global vision system to that used by the
\emph{RoboCup} for the \emph{SSL} games, that can be used as a learning tool for
computer vision and shared memory parallel programming courses.

For a global vision system to be useful as a learning tool, its detection
algorithms must show a clear conceptual division and be implemented simply so it
can be easily modified (even if it implies a performance loss). Using a base
system, we develop a new system able to efficiently use multiple processing
units and its memory hierarchy. Two parallelization strategies where apply.
The first one applies parallelism inside a frame, dividing the frames into
fragments that are processed independently. The other strategy is based on the
simultaneous precessing of different frames.

The system was implemented using \emph{OpenMP} shared memory programming model
for \emph{C++}. The application's parallelization strategies' parameters can be
tuned to obtain maximum performance for a specific hardware platform. On a
server with an Intel Xeon E5-2630 processor (with 6 processing units and
simultaneous multithreading), we achieve an improvement of 5.42$x$ \emph{FPS},
in comparison to the run using only one processing unit (from the 6 available).
The parallelization strategies can be modified, this can be used to led students
experiment and analyze its results to explain its impact on the system
performance.

\vfill
\pagebreak
