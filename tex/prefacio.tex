\ \\
\ \\
\label{pagpref}
\noindent{\LARGE \sc Prefacio}\\
\ \\
\ \\

\ \\

\ \\
\ \\


Esta tesis es presentada como parte de los requisitos finales para optar al
grado académico de {\em Licenciado en Ciencias de la Computación}, otorgado por
la Universidad Nacional del Comahue, y no ha sido presentada previamente para la
obtención de otro título en esta Universidad u otras. La misma es el resultado
de la investigación llevada a cabo en el Departamento de ingenieria de
computadoras de la Facultad de Informática, en el período comprendido entre
febrero de 2015 y mayo de 2016, bajo la dirección de Javier Balladini y la
codirección de Eduardo Grosclaude.

\vspace{3cm}


\ \\
{\flushright Rodrigo S. Cañibano\\
{\sc Facultad de Informática \\
Universidad Nacional del Comahue}\\
{\em Neuquén, 25 de mayo de 2016.}\\}

\vfill

\begin{center}
%
\framebox{\begin{minipage}[t]{0.9\columnwidth}%
\begin{flushleft}
\includegraphics[scale=0.035]{logos/unc.png}

\vspace{-2cm}
{\large \hspace{5cm}\sc universidad
nacional del comahue} \\
\par\end{flushleft}
\begin{center}
{\large \qquad{}}{ \hspace{2.5cm} Facultad de Informática}
\par\end{center}

\vspace{1cm}

\indent \ \ \ \ \ \ \ \ \ \ \ La presente tesis ha sido aprobada el día ........................., mereciendo la \\
\indent \ \ \ \  \ \ \ \ \ calificación de .............................

\medskip{}

\vspace{1cm}
\end{minipage}}
\end{center}

\pagebreak
