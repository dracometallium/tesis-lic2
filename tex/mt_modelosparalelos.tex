% vim: set spell spelllang=es syntax=tex :

\section{Clasificación de los modelos de computo paralelos}

\label{mt_modelosparalelos}

Existen varias formas de clasificar los modelos computacionales, en esta
sección nos enfocaremos en dos clasificaciones que son de particular interés
para éste trabajo: la clasificación de Flynn, que compara la relación entre los
flujos de datos y los de instrucciones, y la clasificación según los modelos de
memoria compartida y distribuida que analiza la forma en las unidades de
procesamiento comparten datos.

\subsection{Taxonomía de Flynn}

La taxonomía de Flynn \cite{flynnstaxonomy1972} propone describir la estructura
general de una arquitectura según la magnitud de interacción de los flujos de
datos e instrucciones, esto permite definir cuatro clases distintas:

\begin{description}

	\item[Single Instruction stream/Single Data stream (\emph{SISD}):] En
		este modelo solo hay un flujo de instrucciones que opera sobre
		un único flujo de datos. Éste modelo se corresponde con la
		arquitectura de Von Neumann.

	\item[Single Instruction stream/Multiple Data stream (\emph{SIMD}):] En
		este modelo solo hay un flujo de instrucciones que opera sobre
		varios flujos de datos al mismo tiempo. Para la ejecución de los
		flujos de instrucciones, las computadoras que implementan este
		modelo tienen varias unidades de procesamiento que comparten la
		misma unidad de control. En un ciclo de instrucción, todas las
		unidades de procesamiento ejecutan la misma instrucción sobre
		distintos elementos de datos. Algunas implementaciones de este
		modelo agregan un bit de actividad asociado a cada unidad de
		procesamiento que indica si debe ejecutar la instrucción o no
		operar. Esta extensión permite implementar estructuras
		condicionales con facilidad pero reduciendo la performance
		\cite{introToPC2002}. Éste modelo es el utilizado en las placas
		aceleradoras modernas.

	\item[Multiple Instruction stream/Single Data stream (\emph{MISD}):] En
		este modelo múltiples flujos de instrucciones operan sobre un
		único flujo de datos. Si todas las unidades de procesamiento
		ejecutan el mismo conjunto de instrucciones, este modelo puede
		ser utilizado con el fin de aumentar la redundancia. Los
		sistemas de vuelo críticos del transbordador espacial STS
		implementaron este modelo con éste propósito
		\cite{spaceShuttlePCS1984}. Una alteración de este modelo consta
		en que solo la primera unidad de procesamiento lea del conjunto
		de datos, mientras que el resto de las unidades tienen como
		datos de entrada los producidos por las anteriores. Las
		\emph{tuberías} de los sistemas \emph{UNIX} pueden verse como
		una implementación de ésta extinción del modelo \emph{MISD}.

	\item[Multiple Instruction stream/Multiple Data stream (\emph{MIMD}):] En
		este modelo múltiples flujos de instrucciones que operan sobre
		múltiples flujos de datos. Al igual que las computadoras
		\emph{SIMD}, tienen múltiples unidades de procesamiento, pero
		cada una con su propia unidad de control, incrementando el costo
		y consumo energético \cite{introToPC2002}. Los sistemas
		multiprocesador de las computadoras de escritorio convencionales
		implementan este modelo.

\end{description}

\subsection{Memoria compartida y memoria distribuida}

Una característica importante de los sistemas \emph{MIMD} es la manera en la
cual las unidades de procesamiento pueden acceder a los datos de las demás,
dependiendo de si todas las unidades de procesamiento comparten el mismo espacio
de direcciones o existe un espacio de direcciones asociado a cada conjunto de
unidades de procesamiento \cite{introToPC2002, anIntroToPP2011}:

\begin{description}

	\item[Memoria compartida:] Solo hay un espacio de memoria para todas las
		unidades. Los hilos de ejecución pueden utilizar memoria
		compartida, permitiendo comunicación implícita.También admiten
		comunicación por paso de mensajes. El espacio de memoria puede
		estar dividido en distintos bancos de memoria asociados a
		distintos conjuntos de unidades de procesamiento. Si el tiempo
		de acceso a los bancos de memoria es independiente de la unidad
		de procesamiento que realiza el acceso, la arquitectura es
		clasificada como \emph{acceso a memoria uniforme}, abreviado
		como \emph{UMA} (del Inglés \emph{Uniform Memory Access}). Si el
		tiempo de acceso depende de la unidad de procesamiento y el
		banco al que se quiere acceder, la arquitectura es clasificada
		como \emph{acceso a memoria no uniforme}, abreviado como
		\emph{NUMA} (del Inglés \emph{Non Uniform Memory Access}). Las
		computadoras multiprocesador de escritorio actuales son sistemas
		de memoria compartida \emph{UMA}.

	\item[Memoria distribuida:] Cada unidad de procesamiento tiene su propio
		banco de memoria y su propio espacio de direcciones. La
		comunicación entre los hilos ejecutando en distintas unidades de
		procesamiento se realiza a través de paso de mensajes, por lo
		que toda la comunicación es explicita y controlada por el
		programador. El acceso a la memoria local es mucho más rápido
		que el acceso a los datos de otra unidad de procesamiento, y
		dependiendo de la red de comunicación puede variar entre
		distintos nodos.

	\item[Sistemas híbridos:] Son sistemas de memoria distribuida donde
		cada unidad de procesamiento es en realidad un sistema de
		memoria compartida.

\end{description}
