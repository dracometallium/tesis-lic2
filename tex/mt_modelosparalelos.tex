% vim: set spell spelllang=es syntax=tex :

\section{Clasificación de los modelos de computo paralelos}

\label{mt_modelosparalelos}

Existen varias formas de clasificar los modelos computacionales, pero en esta
sección nos enfocaremos en dos clasificaciones que son de particular interés
para este trabajo. Por un lado la clasificación de Flynn, que compara la
relación entre los flujos de datos y los de instrucciones, y por otro el modelo
de memoria compartida o distribuida que analiza la forma en las unidades de
procesamiento tienen acceso a la memoria.

\subsection{Taxonomía de Flynn}

La taxonomía de Flynn\cite{flynnstaxonomy1972} propone describir la estructura
general de una computadora según la magnitud de interacción de los flujos de
datos e instrucciones, esto permite definir cuatro clases distintas:

\begin{description}

	\item[Single Instruction stream/Single Data stream (\emph{SISD}):] En
		este modelo solo hay un flujo de instrucciones que opera sobre
		un único flujo de datos. Éste modelo se corresponde con la
		arquitectura de Von Neumann.

	\item[Single Instruction stream/Multiple Data stream (\emph{SIMD}):] En
		este modelo solo hay un flujo de instrucciones que opera sobre
		varios flujos de datos al mismo tiempo. Para la ejecución de los
		flujos de instrucciones, las computadoras que implementan este
		modelo tienen varias unidades de procesamiento que comparten la
		misma unidad de control. En un ciclo de instrucción, todas las
		unidades de procesamiento ejecutan la misma instrucción sobre
		distintos elementos de datos. Algunas implementaciones de este
		modelo agregan un bit de actividad asociado a cada unidad de
		procesamiento que indica si debe ejecutar la instrucción o no
		operar. Esta extensión permite implementar estructuras
		condicionales con facilidad pero reduciendo la performance
		\cite{introToPC2002}. Éste modelo es el utilizado en las placas
		aceleradoras modernas.

	\item[Multiple Instruction stream/Single Data stream (\emph{MISD}):] En
		este modelo múltiples flujos de instrucciones operan sobre un
		único flujo de datos. Si todas las unidades de procesamiento
		ejecutan el mismo conjunto de instrucciones, este modelo puede
		ser utilizado con el fin de aumentar la redundancia. Los
		sistemas de vuelo críticos del transbordador espacial STS
		implementaron este modelo con éste propósito
		\cite{spaceShuttlePCS1984}. Una alteración de este modelo consta
		en que solo la primera unidad de procesamiento lea del conjunto
		de datos, mientras que el resto de las unidades tienen como
		datos de entrada los producidos por las anteriores. Las
		\emph{tuberías} de los sistemas \emph{UNIX} pueden verse como
		una implementación de ésta extinción del modelo \emph{MISD}.

	\item[Multiple Instruction stream/Multiple Data stream (\emph{MIMD}):] En
		este modelo múltiples flujos de instrucciones que operan sobre
		múltiples flujos de datos. Al igual que las computadoras
		\emph{SIMD}, tienen múltiples unidades de procesamiento, pero
		cada una con su propia unidad de control, incrementando el costo
		y consumo energético \cite{introToPC2002}. Los sistemas
		multiprocesador de las computadoras de escritorio convencionales
		implementan este modelo.

\end{description}

\subsection{Memoria compartida y memoria distribuida}

bla bla bla
