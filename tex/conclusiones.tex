% vim: set spell spelllang=es syntax=tex :

\section{Conclusiones finales}

Durante el desarrollo de esta tesis se logro cumplir satisfactoriamente con los
objetivos de esta. El software desarrollado es capas de procesar un vídeo de
800x600 píxeles de resolución a una taza de 195 cuadros por segundo, más del 3
veces lo buscado originalmente. Ademas se pudo observar y analizar como se
comportaba el sistema bajo distintas configuraciones. Se pudo comprobar como un
numero primo de fragmentos afectaba de forma negativa el desempeño del sistema,
y como un bajo numero de estos obliga a los hilos a competir por los recursos de
caché.

El framework final es adaptable a distintos dominios y su arquitectura basada en
plugins permite modificarlos y agregar nuevos fácilmente. Finalmente,
modificando las variables de cantidad de fragmentos e hilos de búsqueda el
sistema puede adaptarse a un amplio rango de configuraciones de hardware
fácilmente.

\section{Trabajos futuros}

Si bien el sistema funciona correctamente, existen aspectos que hacen difícil su
aplicación como remplazo para los sistemas de visión global utilizados hoy en
día en las competencias de la \emph{SSL}. En primer lugar, actualmente el plugin
de difusión notifica la posición de los robots con respecto a las coordenadas
del fragmento, en lugar de utilizar las de la imagen original. En segundo lugar,
en esta versión, si se desea calibrar los colores de los parches o la pelota se
deben realizar modificaciones en el código. Seria deseable una interfaz gráfica
que permita modificar estos datos interactivamente. Seria útil también mostrar
gráficamente la detección de los objetos.
