% vim: set spell spelllang=es syntax=tex :

\section{Conclusiones finales}

\label{concluciones}

En este trabajo se ha presentado un nuevo sistema de visión global por
computadora para el fútbol de robots de la \emph{SSL}, que puede ser utilizado
como herramienta educativa en asignaturas de visión por computadora y sistemas
paralelos, y permite explorar distintas estrategias de paralelización. El
sistema procesa cuadros de video y reporta la posición y orientación de los
robots y la posición de la pelota en cada uno de ellos.

Se describieron dos estrategias de paralelización que son aplicadas en conjunto.
Una de ellas explota el paralelismo dentro de cada cuadro, dividiendo los
cuadros en fragmentos que son procesados de forma independiente. La otra
estrategia se basa en el procesamiento simultáneo de diferentes cuadros del
video (independientes unos de otros).

Se realizó una implementación utilizando el modelo de programación de memoria
compartida \emph{OpenMP} para \emph{C++}, basada en plugins para facilitar su
modificación en un ambiente educativo. Con el fin de sintonizar la aplicación
para extraer el máximo rendimiento de una determinada plataforma hardware, el
sistema cuenta con diferentes parámetros que permiten modificar el
comportamiento de sus estrategias de paralelización. La posibilidad de modificar
el comportamiento de las estrategias de paralelización es útil para que el
estudiante realice experimentación y analice los resultados buscando
explicaciones al impacto en el rendimiento del sistema.

Se estudió el comportamiento del sistema en términos de \emph{FPS} y de retardo
en el procesamiento de los cuadros para diferentes configuraciones del sistema
con videos de distintas resoluciones. Se observó que dividir el cuadro
en un número primo de fragmentos afectaba de forma negativa el desempeño del
sistema, y que un bajo número de fragmentos obliga a los hilos a competir por
los recursos de caché. En un servidor con un procesador Intel Xeon E5-2630 (6
núcleos y multithreading simultáneo) el software es capaz de procesar un video
de 800x600 píxeles de resolución a una taza de 196 cuadros por segundo y un
retardo de procesamiento del cuadro de 318$ms$; y, un video de 1280x720 píxeles
resolución a una taza de 130 cuadros por segundo y un retardo de procesamiento
del cuadro de 113$ms$. Se logro una mejora de 5,42$x$ en los cuadros procesados
por segundo, con respecto a la ejecución del sistema utilizando un único núcleo
(de los 6 disponibles). Esto demuestra que el sistema escala adecuadamente.

Se propusieron ejercicios prácticos que utilizan el sistema como herramienta
didáctica para la enseñanza de visión por computadora y programación de sistemas
paralelos. Ellos se enfocan principalmente en análisis de rendimiento
(\emph{speedup} y eficiencia), impacto de la jerarquía de memoria,
particionamiento de los datos, y programación de plugins y pilas de plugins para
visión por computadora.

\section{Trabajos futuros}

\label{trabajosFuturos}

Si bien el sistema funciona correctamente, existen aspectos que hacen difícil su
aplicación como remplazo para los sistemas de visión global utilizados hoy en
día en las competencias de la \emph{SSL}. En la implementación actual, si se
desea calibrar los colores de los parches o la pelota se deben realizar
modificaciones en el código. Seria deseable una interfaz gráfica que permita
modificar estos datos interactivamente. Seria útil también mostrar gráficamente
la detección de los objetos.

La taza de crecimiento del \emph{speedup} se vio disminuida a partir de los cinco
hilos de búsqueda. Posibles causas son que los accesos a memoria crean un cuello
de botella, o que con cinco hilos de búsqueda se alcanza un alto paralelismo a
nivel de instrucción. Para investigar estas hipótesis podría analizarse la
cantidad de instrucciones por ciclo en cada caso.

Además, se cree conveniente implementar soluciones paralelas optimizadas de los
plugins.

