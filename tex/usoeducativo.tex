% vim: set spell spelllang=es syntax=tex :

\section{Uso como herramienta educativa}

\label{usoEducativo}

El sistema puede ser utilizado como herramienta educativa en varios niveles, y
en materias de visión por computadora como materias de sistemas paralelos.

El uso mas sencillo que se le puede dar al sistema es simplemente como sistema
de visión global para torneos de fútbol de robots. Dado que ya existe el sistema
utilizado por la \emph{SSL} (llamado \emph{SSL-Vision}), el nuevo sistema solo
sera útil cuando se necesite un balance especifico de \emph{FPS} y retardo de
procesamiento.

En una materia de sistemas paralelos podría utilizarse el sistema para que los
alumnos experimenten como el varia el rendimiento bajo distintas cantidades de
hilos de búsquedas y particiones en diferente hardware, introduciéndose de esta
manera en las ventajas y limitaciones de los sistemas paralelos. Luego podría
pedírsele a los alumnos que realicen predicciones sobre como se comportara el
sistema bajo nuevas configuraciones o que busquen configuraciones que cumplas
con requerimientos específicos de retardo máximo y \emph{FPS} mínimos.

Los plugins del sistema pueden ser modificados tanto en materias de visión por
computadora como en las de sistemas paralelos, pero con enfoques distintos. En
las primeras los alumnos podrían realizar como practica el mejoramiento de la
precisión los plugins, mientras que en la segunda se podría solicitar a los
alumnos que reduzcan los tiempos de los plugins aplicando paralelismo a nivel de
instrucción.

Las materias de visión por computadora más avanzadas podrían tener como trabajo
final la implementación de nuevos mecanismos para detectar los robots, pelota, o
nuevos objetos, esto implicaría la creación de nuevos plugins y pilas de
plugins. También se puede plantear como trabajo final adaptar el framework para
un dominio distinto.
