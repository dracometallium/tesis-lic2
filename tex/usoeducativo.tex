% vim: set spell spelllang=es syntax=tex :

En este capitulo sugeriremos un conjunto de actividades en las cuales el sistema
propuesto en esta tesis puede ser utilizado como herramienta educativa tanto en
materias de visión por computadora como en materias de sistemas paralelos.

\section{Actividades para una materia de sistemas paralelos}

\label{eduparalelos}

En una materia de sistemas paralelos el sistema puede cumplir varios objetivos.
Comenzando con el calculo de \emph{speedup} y eficiencia, hasta la aplicación de
paradigmas de programación específicos con el fin de mejorar su rendimiento.

\subsection{Actividad sugerida 1}

Una vez introducidos los conceptos de \emph{speedup} y eficiencia, el sistema
puede ser utilizado como plataforma para la realización de ejercicios prácticos
de cálculo de dichas magnitudes.

\begin{description}

	\item{Conocimientos previos}: Introducción a los conceptos de
		\emph{speedup} y eficiencia, Arquitectura de computadoras
		paralelas.

	\item{Objetivos}: Reforzar conceptos de \emph{speedup} y eficiencia.

	\item{Consigna}: \begin{enumerate}

	\item{Utilizando el sistema para procesar el video de 800x600 píxeles,
		dividiéndolo en 4 fragmentos:

\begin{enumerate}

	\item{¿Cuántos cuadros por segundo se logran procesar si se utilizan 1
		hasta 12 hilos de búsqueda?}

	\item{Calcule el \emph{speedup} y la \emph{eficiencia} cuando se
		utilizan 2 hilos de búsqueda.}

\end{enumerate}}

\end{enumerate}

\end{description}

\subsection{Actividad sugerida 2}

Luego de la actividad sugerida 1, y con el fin de reforzar el concepto de
\emph{speedup} lineal y no lineal, se puede solicitar a los alumnos que comparen
la curva de \emph{speedup} del sistema contra la curva del \emph{speedup} ideal.
Luego de contrastar las curvas, el alumno, utilizando sus conocimientos de
arquitectura de computadoras paralelas, podría intentar encontrar las causas de
la reducción del crecimiento del rendimiento a medida que el sistema escala.

\begin{description}

	\item{Conocimientos previos}: Introducción al concepto de
		\emph{speedup}, eficiencia real e ideal, Arquitectura de
		computadoras paralelas.

	\item{Objetivos}: Reforzar diferencias entre eficiencia real e ideal.

	\item{Consigna}: \begin{enumerate}

	\item{[...]

\begin{enumerate}

	\setcounter{enumii}{2}

	\item{Compare los resultados obtenidos en el inciso anterior con el
		\emph{speedup} ideal ¿Qué explicaciones posibles existen para la
		reducción del rendimiento?}

	\item{¿Cuál espera que sea el comportamiento del sistema si se utilizan
		128 hilos de búsqueda? ¿Y si se utilizan 1024?}

	\item{Si se tiene una computadora con un procesador de 4 núcleos
		¿Cuántos hilos de búsqueda recomendaría?}

\end{enumerate}}

\end{enumerate}

\end{description}

\subsection{Actividad sugerida 3}

Como se pudo verificar en este trabajo, la fragmentación de los datos tiene un
impacto crucial en la eficiencia de los programas paralelos. Para que los
alumnos experimenten con este concepto el sistema desarrollado es ideal, ya que
permite experimentar con distintas cantidad de hilos y fragmentos. Para que los
alumnos puedan fortalecer este concepto proponemos como práctica que los
estudiantes ejecuten el programa dividiendo el cuadro en 16, 17 y 18 partes, y
luego busquen otras cantidades de fragmentos donde se produzca una caída de los
\emph{FPS} con respecto a las cantidades de fragmentos inmediatas. Con esto se
espera que los alumnos puedan concluir que las cantidades de fragmentos sin
divisores propios (como los números primos) son perjudiciales para la
performance, y de esta manera reforzar la idea de que la distribución de los
datos es importante.

\begin{description}

	\item{Conocimientos previos}: Distribución de datos en arquitectura de
		computadoras paralelas.

	\item{Objetivos}: Enfatizar la importancia de la correcta distribución
		de los datos en arquitecturas paralelas.

	\item{Consigna}: \begin{enumerate}
	
	\setcounter{enumi}{1}

	\item{Si se utilizan 4 hilos de búsqueda y se dividen los cuadros en 16,
		17 y 18 partes para procesar el video de 1280x720 píxeles:

\begin{enumerate}

	\item{¿Cuántos cuadros por segundo procesa el sistema?}

	\item{Limitándose a una cantidad de fragmentos menor a 17 ¿Encuentra
		otros valores para los cuales el sistema tenga mayor rendimiento
		tanto si se utiliza un fragmento menos o un fragmento más?}

\end{enumerate}}

\end{enumerate}

\end{description}

\subsection{Actividad sugerida 4}

Una vez que los alumnos tienen incorporados los conceptos de \emph{speedup},
eficiencia, y la importancia de la distribución de los datos, proponemos que
experimenten con distintas configuraciones de ejecución con el fin de que el
sistema cumpla con requerimientos específicos de tiempo de procesamiento y
\emph{FPS}. Con esto se espera que los estudiantes incorporen la noción de que
distintas aplicaciones pueden tener distintos requerimientos específicos que
deben ser tenidos en cuenta al momento de configurarla.

\begin{description}

	\item{Conocimientos previos}: Arquitectura de computadoras paralelas.

	\item{Objetivos}: Establecer requerimientos y configurar una aplicación
		acorde a ellos.

	\item{Consigna}: \begin{enumerate}

	\setcounter{enumi}{2}

	\item{Utilizando el sistema para procesar el video de 800x600 píxeles:

\begin{enumerate}

	\item{Si se utilizan 12 hilos de búsqueda ¿En cuantos fragmentos deberán
		dividirse los cuadros para que la cantidad de cuadros por
		segundo procesados sea mayor a 180?}

	\item{Si se utilizan 12 hilos de búsqueda ¿En cuantos fragmentos deberán
		dividirse los cuadros para que el retardo del cuadro sea menor a
		6 centésimas de segundo?}

	\item{Si se divide el cuadro en 15 fragmentos ¿Cuántos hilos de búsqueda
		se deberán usar para que la cantidad de cuadros por segundo
		procesados sea mayor a 180?}

\end{enumerate}}

\end{enumerate}

\end{description}

\subsection{Actividad sugerida 5}

Finalmente si se desea utilizar el sistema con la finalidad de que los alumnos
realicen práctica de programación de sistemas paralelos, se puede plantear como
ejercicio la optimización del plugin de \emph{segmentación de color} utilizando
instrucciones vectoriales.

\begin{description}

	\item{Conocimientos previos}: Instrucciones vectoriales, Arquitectura de
		computadoras paralelas.

	\item{Objetivos}: Desarrollar competencias practicas de programación
		utilizando el paradigma de paralelismo de datos.

	\item{Consigna}: \begin{enumerate}

	\setcounter{enumi}{3}

	\item{Modifique el plugin de \emph{segmentación de color} utilizando
	instrucciones vectoriales ¿Cuál es la mejora obtenida?}

\end{enumerate}

\end{description}

\section{Actividades para una materia de visión por computadora}

\label{eduvision}

Para las materias de visión por computadora se puede hacer uso de la facilidad
de extensión del sistema. Ya sea extendiendo el dominio del sistema o
redefiniéndolo completamente.

\subsection{Actividad sugerida 1}

Los alumnos pueden usar el sistema como base para la la implementación de nuevos
mecanismos para detectar los robots, pelota, o nuevos objetos, esto implicaría
la creación de nuevos plugins y pilas de plugins.

\begin{description}

	\item{Conocimientos previos}: Detección y Segmentación.

	\item{Objetivos}: Obtener habilidades practicas en el diseño e
		implementación de plugins para de aplicaciones de visión por
		computadora. Fortalecer los conceptos de detección y
		segmentación.

	\item{Consigna}: \begin{enumerate}

	\item{Para incrementar la detección de infracciones en los partidos de
		la \emph{SSL} se desea agregar un referí a la cancha. Éste es un
		robot similar a los jugadores que proveerá una vista a nivel del
		suelo, se moverá por toda la cancha, y lleva un parche
		cuadriculado blanco y negro. Implemente una nueva pila de
		plugins para la detección del referí.}

\end{enumerate}

\end{description}

\subsection{Actividad sugerida 2}

También se puede plantear como trabajo adaptar el framework para un dominio
distinto.

\begin{description}

	\item{Conocimientos previos}: Etapas de un sistema de visión por
		computadora.

	\item{Objetivos}: Obtener habilidades practicas en el diseño e
		implementación de aplicaciones de visión por computadora.
		Consolidar conceptos de todas las etapas de los sistemas de
		visión por computadora.

	\item{Consigna}: \begin{enumerate}

	\setcounter{enumi}{1}

	\item{Utilizando como base el framework, implemente un sistema capaz de
		detectar la presencia de un marcador de realidad aumentada, su
		posición y orientación con respecto a la cámara, y lo reemplace
		por el logo de la \texttt{Universidad Nacional del Comahue} (con
		la orientación correspondiente). El video resultante debe tener
		un retardo menor a un segundo.  Puede diseñar su propio marcador
		de realidad aumentada, aunque se recomienda utilizar uno
		estándar.}

\end{enumerate}

\end{description}
