% vim: set spell spelllang=es syntax=tex :

\section{Uso como herramienta educativa}

\label{usoEducativo}

El sistema puede ser utilizado como herramienta educativa en varios niveles, y
tanto en materias de visión por computadora como en materias de sistemas
paralelos.

En una materia de sistemas paralelos el sistema puede cumplir varios objetivos.
Una vez introducidos los conceptos de \emph{speedup} y eficiencia el sistema
puede ser utilizado para que los alumnos puedan practicar su cálculo.

\begin{enumerate}

	\item{Utilizando el sistema para procesar el video de 800x600 píxeles,
		dividiéndolo en 4 fragmentos:

\begin{enumerate}

	\item{¿Cuantos cuadros por segundo se logran procesar si se utilizan 1
		hasta 12 hilos de búsqueda?}

	\item{Calcule el \emph{speedup} y la \emph{eficiencia} cuando se
		utilizan 2 hilos de búsqueda.}

\end{enumerate}}

\end{enumerate}

Con el fin de reforzar el concepto de \emph{speedup} lineal y no lineal, se
puede solicitar a los alumnos que comparen la curva de \emph{speedup} del
sistema contra la curva del \emph{speedup} ideal. Luego de contrastar las
curvas, el alumno, utilizando sus conocimientos de arquitectura de computadoras
paralelas, podría intentar encontrar las causas de la reducción del crecimiento
del rendimiento a medida que el sistema escala.

\begin{enumerate}

	\item{[...]

\begin{enumerate}

	\setcounter{enumii}{2}

	\item{Compare los resultados obtenidos en el inciso anterior con el
		\emph{speedup} ideal ¿Que explicaciones posibles existen para la
		reducción del rendimiento?}

	\item{¿Cual espera que sea el comportamiento del sistema si se utilizan
		128 hilos de búsqueda? ¿Y si se utilizan 1024?}

	\item{Si se tiene una computadora con un procesador de 4 núcleos
		¿Cuantos hilos de búsqueda recomendaría?}

\end{enumerate}}

\end{enumerate}

Como se pudo verificar en este trabajo, la fragmentación de los datos tiene un
impacto crucial en la eficiencia de los programas paralelos. Para que los
alumnos experimenten con este concepto el sistema desarrollado es ideal, ya que
permite experimentar con distintas cantidad de hilos y cantidad de fragmentos.
Para que los alumnos puedan fortalecer este concepto proponemos como práctica
que los estudiantes ejecuten el programa dividiendo el cuadro en 16, 17 y 18
partes, y luego busquen otras cantidades de divisiones donde se produzca una
caída de los \emph{FPS} con respecto a las cantidades de fragmentos inmediatas.
Con esto se espera que los alumnos puedan concluir que las cantidades de
divisiones cuyos valores son primos son perjudiciales y de esta manera
reforzar la idea de que la distribución de los datos es importante.

\begin{enumerate}
	
	\setcounter{enumi}{1}

	\item{Si se utilizan 4 hilos de búsqueda y se dividen los cuadros en 16,
		17 y 18 partes para procesar el video de 1280x720 píxeles:

\begin{enumerate}

	\item{¿Cuantos cuadros por segundo procesa el sistema?}

	\item{Limitándose a una cantidad de fragmentos menor a 17 ¿Encuentra
		otros valores para los cuales el sistema tenga mayor rendimiento
		tanto si se utiliza un hilo menos o un hilo más?}

\end{enumerate}}


\end{enumerate}

Una vez que los alumnos tienen incorporados los conceptos de \emph{speedup},
eficiencia, y la importancia de la distribución de los datos, proponemos que
experimenten con distintas configuraciones de ejecución con el fin de que el
sistema cumpla con requerimientos específicos de tiempo de procesamiento y
\emph{FPS}. Con ésto se espera que los estudiantes incorporen la noción de que
distintas aplicaciones pueden tener distintos requerimientos específicos que
deben ser tenidos en cuenta al momento de configurarla.

\begin{enumerate}

	\setcounter{enumi}{2}

	\item{Utilizando el sistema para procesar el video de 800x600 píxeles:

\begin{enumerate}

	\item{Si se utilizan 12 hilos de búsqueda ¿En cuantos fragmentos deberán
		dividirse los cuadros para que la cantidad de cuadros por
		segundo procesados sea mayor a 180?}

	\item{Si se utilizan 12 hilos de búsqueda ¿En cuantos fragmentos deberán
		dividirse los cuadros para que el retardo del cuadro sea menor a
		6 centésimas de segundo?}

	\item{Si se divide el cuadro en 15 fragmentos ¿Cuantos hilos de búsqueda
		se deberán usar para que la cantidad de cuadros por segundo
		procesados sea mayor a 180?}

\end{enumerate}}

\end{enumerate}

Finalmente si se desea utilizar el sistema con la finalidad de que los alumnos
realicen práctica de programación de sistemas paralelos, se puede plantear como
ejercicio la optimización del pluguin de \emph{segmentación de color} utilizando
instrucciones escalares.

\begin{enumerate}

	\setcounter{enumi}{3}

	\item{Modifique el plaguin de \emph{segmentacion de color} utilizando
	instrucciones vectoriales ¿Cual es la mejora obtenida?}

\end{enumerate}

Para las materias de visión por computadora se puede hacer uso de la facilidad
de extensión del sistema. Los alumnos pueden usar el sistema como base para la
la implementación de nuevos mecanismos para detectar los robots, pelota, o
nuevos objetos, esto implicaría la creación de nuevos plugins y pilas de
plugins.

\begin{enumerate}

	\item{Para incrementar la detección de infracciones en los partidos de
		la \emph{SSL} se desea agregar un referí a la cancha. Éste es un
		robot similar a los jugadores que proveerá una vista a nivel del
		suelo, se moverá por toda la cancha, y lleva un parche
		cuadriculado blanco y negro. Implemente una nueva pila de
		plugins para la detección del referí.}

\end{enumerate}

También se puede plantear como trabajo adaptar el framework para un dominio
distinto.

\begin{enumerate}

	\item{Utilizando como base el framework, implemente un sistema capaz de
		detectar la presencia un marcador de realidad aumentada, su
		poción y orientación con respecto a la cámara, y lo remplace por
		el logo de la \texttt{Universidad Nacional del Comahue} (con la
		orientación correspondiente). El video resultante debe tener un
		retardo menor a un segundo. Puede diseñar su propio marcador de
		realidad aumentada, aunque se recomienda utilizar uno estándar.}

\end{enumerate}
