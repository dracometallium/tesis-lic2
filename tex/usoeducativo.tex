% vim: set spell spelllang=es syntax=tex :

\section{Uso como herramienta educativa}

\label{usoEducativo}

El sistema puede ser utilizado como herramienta educativa en varios niveles, y
en materias de visión por computadora como materias de sistemas paralelos.

El uso más sencillo que se le puede dar al sistema es simplemente como sistema
de visión global para torneos de fútbol de robots. Dado que ya existe el sistema
utilizado por la \emph{SSL} (llamado \emph{SSL-Vision}), el nuevo sistema solo
sera útil cuando se necesite un balance especifico de \emph{FPS} y retardo de
procesamiento.

En una materia de sistemas paralelos podría utilizarse el sistema para que los
alumnos experimenten como el varia el rendimiento bajo distintas cantidades de
hilos de búsquedas y particiones en diferente hardware, introduciéndose de esta
manera en las ventajas y limitaciones de los sistemas paralelos. Para este fin
se podrían utilizar las preguntas presentadas a continuación. Algunos valores
deberán ajustadas dependiendo de la plataforma experimental con la que cuenten
los alumnos, en estos ejemplos se considerara que cuentan con la misma
plataforma experimental que de este trabajo.

\begin{enumerate}

	\item{Utilizando el sistema para procesar el vídeo de 800x600 píxeles,
		dividiéndolo el 4 fragmentos:

\begin{enumerate}

	\item{¿Cuantos cuadros por segundo se logran procesar si se utilizan 1 y
		2 hilos de búsqueda?}

	\item{¿Cuantos cuadros por segundo estima que se procesaran
		aproximadamente si se utilizan 3, 4 y 16 hilos de búsqueda?}

	\item{Realice el experimento de procesar el vídeo con 3, 4 y 16 hilos de
		búsqueda ¿Que tan precisa fueron sus predicciones realizadas en
		el inciso anterior?}

	\item{¿Cual espera que sea el comportamiento del sistema si se utilizan
		128 hilos de búsqueda? ¿Y si se utilizan 1024?}

	\item{¿Que explicaciones posibles existen para la reducción del
		\emph{speedup}?}

	\item{Si se tiene una computadora con un procesador de 4 núcleos
		¿Cuantos hilos de búsqueda recomendaría?}

\end{enumerate}}

	\item{Si se utilizan 4 hilos de búsqueda y se dividen los cuadros en 16,
		17 y 18 partes para procesar el vídeo de 800x600 píxeles:

\begin{enumerate}

	\item{¿Cuantos cuadros por segundo procesa el sistema?}

	\item{Limitándose a una cantidad de hilos de búsqueda menor a 17
		¿Encuentra otros valores para los cuales el sistema tenga mayor
		rendimiento tanto si se utiliza un hilo menos o un hilo más?}

\end{enumerate}}

	\item{Utilizando el sistema para procesar el vídeo de 800x600 píxeles:

\begin{enumerate}

	\item{Si se utilizan 12 hilos de búsqueda ¿En cuantos fragmentos deberán
		dividirse los cuadros para que la cantidad de cuadros por
		segundo procesados sea mayor a 180?}

	\item{Si se utilizan 12 hilos de búsqueda ¿En cuantos fragmentos deberán
		dividirse los cuadros para que el retardo del cuadro sea menor a
		6 centésimas de segundo?}

	\item{Si se divide el cuadro en 15 fragmentos ¿Cuantos hilos de búsqueda
		se deberán usar para que la cantidad de cuadros por segundo
		procesados sea mayor a 180?}

\end{enumerate}}

\end{enumerate}

Los plugins del sistema pueden ser modificados tanto en materias de visión por
computadora como en las de sistemas paralelos, pero con enfoques distintos. En
las primeras los alumnos podrían realizar como practica el mejoramiento de la
precisión los plugins, mientras que en la segunda se podría solicitar a los
alumnos que reduzcan los tiempos de los plugins aplicando paralelismo a nivel de
instrucción.

Las materias de visión por computadora más avanzadas podrían tener como trabajo
final la implementación de nuevos mecanismos para detectar los robots, pelota, o
nuevos objetos, esto implicaría la creación de nuevos plugins y pilas de
plugins. También se puede plantear como trabajo final adaptar el framework para
un dominio distinto.
