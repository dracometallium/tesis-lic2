% vim: set spell spelllang=es syntax=tex :

\section{Descripción del sistema de visión global para fútbol de robots
físicos sobre el que se basa la presente propuesta}
\sectionmark{Descripción del sistema de visión global para fútbol de robots
físicos...}

El sistema descripto en \cite{torres2014} tiene como objetivo ser utilizado como
herramienta didáctica para la introducción a la visión por computadora. Éste
sistema esta basado en pilas de plugins y desacopla el hilo de captura del los
hilos de búsqueda. Se utilizan múltiples hilos para aprovechar hasta cuatro
núcleos (para el fútbol de robots), pero si el sistema pose mas núcleos estos no
son utilizados. El sistema define un framework de visión por computadora capas
de adaptarse a distintos dominios, pero se provee una solución especifica para
el fútbol de robots de tamaño pequeño. El framework general tiene tres
componentes principales:

\begin{description}

	\item[Hilo principal:] es el hilo encargado de la interfaz gráfica de
		usuario.
	
	\item[Hilo de captura de cuadros:] es el hilo encargado de la captura o
		generación de los cuadros a partir de una cámara o archivo.

	\item[Hilos de procesamiento de cuadros:] estos hilos están formados por
		una pila de plugins que procesaran cada cuadro en forma
		secuencial. Normalmente, por cada tipo de objeto a reconocer,
		hay un hilo de este tipo dedicado a su detección. La diferencia
		entre cada uno de ellos esta en los plugins que componen la pila
		y en su orden. Existen mecanismos de control para asegurar que
		si un cuadro es procesado por hilo de procesamiento entonces
		sera procesado por el resto de los hilos de procesamiento,
		mientras que al mismo tiempo se asegura que cada cuadro se
		procese solo una ves por cada hilo.

\end{description}

En la implementación de la solución especifica para el fútbol de robots de la
\emph{SSL}, se crean dos hilos de
procesamiento de cuadros, uno para la búsqueda de los robots y la otro para la
búsqueda de la pelota. Las primeras etapas de procesamiento son iguales en
ambos. Los primeros plugins son los de conversión de color, segmentación de
color y morfología. El hilo de procesamiento que busca la pelota continua con el
plugin de búsqueda de pelota, mientras que aquel encargado de la búsqueda de
robots continua con los plugins de detección de regiones principales y detección
de detecciones secundarias. Ambos terminan con el plugin de difusión. De
acuerdo a su función, los plugins pueden ser agrupados por las etapas de la
visión por computadora que implementan:

\begin{description}

\item[Adquisicion de la imagen:] Esta etapa esta a cargo del hilo de captura de
	cuadros.

\item[Pre procesamiento:] plugins de conversión de color, de segmentación de
	color y de morfología.

\item[Extraccion de características, Detección y Segmentación:] plugins de
	detección de regiones principales.

\item[Procesamiento de alto nivel:] plugins de detección de pelota y de
	detección de regiones secundarias.

\item[Toma de decisiones:] Dado que el sistema de visión para fútbol de robots
	no realiza toma de decisiones, el estado de la cancha es comunicado a
	las computadoras de los equipos a través del plugin de difusión.

\end{description}

La generación de los cuadros y su procesamiento son independientes, y los
tiempos de generación y procesamiento pueden ser distintos y variables, por lo
cual es necesario un buffer entre el hilo de captura de cuadros y los hilos de
procesamiento de cuadros. En caso de que el buffer alcance su capacidad máxima,
se descartan los cuadros mas viejos.
