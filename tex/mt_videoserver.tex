% vim: set spell spelllang=es syntax=tex :

\section{Detalles del sistema de visión global para fútbol de robots físicos por
Guillermo Torres}

Como se dijo con anterioridad, el objetivo principal de un sistema de visión por
computadora es el de definir un modelo a partir de una escena. En el fútbol de
robots el modelo que se intenta definir es la ubicación y orientación de los
robots y pelota con respecto a la cancha.

Como ya se explico, físicamente, el sistema de visión global para fútbol de
robots consta de un conjunto de cámaras posicionadas sobre la cancha, una
computadora sobre la que corre el software de visión global y la conexión de red
hacia las computadoras de los equipos. Lo que aun no hemos explicado es la
estructura interna del software.

Si bien pueden existir variantes entre los distintos sistemas de visión que se
han propuesto para el fútbol de robots, la mayoría de las soluciones convergen a
un sistema de pila de plugins. Incluso los plugins utilizados
suelen ser similares. Esto no debería ser sorpresivo, ya que el dominio es muy
especifico y el ambiente esta preparado para facilitar la extracción de la
información por el sistema.

El sistema desarrollado por Torres tiene cuatro componentes principales:

\begin{description}

\item[Interfaz de captura:] Interfaz para definir la clase que se encarga de
	obtener los cuadros del vídeo, ya sea de un archivo, cámara u otra
	fuente.

\item[Buffer circular:] Un buffer circular donde se coloca cada uno de los
	cuadros obtenidos.

\item[Estrategia de búsqueda:] Es una interfaz para definir la clase que
	especificara la estrategia de búsqueda, es decir, es el componente
	encargado de controlar de que manera se entregan los cuadros a la pila
	de plugins (comúnmente verificando de no entregar dos veces el
	mismo cuadro a la pila). Cada instancia tiene una sola pila de
	plugins.
	
\item[Pila de plugins:] Define el método \emph{process} que recibe una imagen
	como parámetro. En su creación se le definen los plugins que
	deben procesar la imagen y en que orden. La imagen se procesara de forma
	secuencial.

\item[Plugin:] Interfaz para definir las clases de los plugins que
	procesaran la imagen. Define el método \emph{process} que recibe como
	parámetro una imagen.

\end{description}

En la implementación de la solución especifica, se crean dos estrategias de
búsqueda, una para la búsqueda de los robots y la otra para la búsqueda de la
pelota. Las primeras etapas de procesamiento son iguales en ambas estrategias de
búsqueda. Los primeros plugins son los de calibración, conversión de
color y morfología. La estrategia de búsqueda de pelota continua con el
plugin de búsqueda de pelota, mientras que la búsqueda de robots continua
con los plugins de detección de regiones principales y detección de
detecciones secundarias. Ambas terminan con el plugin de difusión.

TODO: Conectar de alguna manera!!!

Los plugins pueden ser agrupados de acuerdo con las etapas de la visión
por computadora discutidos en la sección anterior:

\begin{description}

\item[Adquisicion de la imagen:] Esta etapa esta a cargo de la interfaz de
	captura.

\item[Pre procesamiento:] plugins de calibración y conversión de color.

\item[Extraccion de características, Detección y Segmentación:] plugins
	de segmentación de color, de morfología y de detección de regiones
	principales.

\item[Procesamiento de alto nivel:] plugins de detección de pelota y de
	detección de regiones secundarias.

\item[Toma de decisiones:] Dado que el sistema de visión para fútbol de robots
	no realiza toma de decisiones, el estado de la cancha es comunicado a
	las computadoras de los equipos a través del plugin de difusión.

\end{description}
