% vim: set spell spelllang=es syntax=tex :

\section{Sistema de visión global para fútbol de robots físicos}

El software de visión global utilizado en la \emph{SSL} de la \emph{Robocup} es
el \emph{SSL-Vision}. Este es un sistema programado en \emph{C++} basado en
\emph{plugins}, lo que le permite ser extendido mediante estos. Lamentablemente
posee algunas desventajas. En primer lugar, el sistema de captura de cuadros y
el de procesamiento de estos están fuertemente acoplados y forman parte del
mismo hilo de ejecución, lo que impide hacer uso de las capacidades de
paralelización que ofrece el hardware actual. En segundo lugar, la complejidad
del sistema entorpece su uso como herramienta didáctica para la introducción a
la visión por computadora.

En respuesta a estos problemas, Guillermo Torres\cite{torres2014} diseñó un
nuevo framework, también programado en \emph{C++} y orientado a \emph{plugins}
destinado al uso educativo y de producción. El framework es general, pero se
provee una implementación de un sistema de visión de fútbol de robots.

Los componentes principales del framework son los siguientes:

\begin{description}

	\item[CaptureInterface:] Interfaz para definir la clase que se encarga
		de obtener los cuadros del vídeo, ya sea de un archivo, cámara u
		otra fuente.

	\item[RingBuffer:] Un buffer circular donde se coloca cada uno de los
		cuadros obtenidos.

	\item[TrackingStrategy:] Es una interfaz para definir la clase que
		especificara la estrategia de búsqueda, es decir, es la clase
		encargada de controlar de que manera se entregan los cuadros a
		la pila de plugins(comúnmente verificando de no entregar dos
		veces el mismo cuadro a la pila). Cada \emph{TrackingStrategy}
		tiene una sola pila de plugins.
	
	\item[VisionStack:] Pila de plugins. Define el método \emph{process} que
		recibe una imagen como parámetro. En su creación se le definen
		los \emph{plugins} que deben procesar la imagen y en que orden.
		La imagen se procesara de forma secuencial.

	\item[VisionPlugin:] Interfaz para definir las clases de los plugins.
		Define el método \emph{process} que recibe como parámetro una
		imagen.

\end{description}

En la solución especifica para fútbol de robots se implementaron dos métodos de
captura (desde un archivo y desde la cámara). Se definió una clase llamada
\emph{GlobalTracking} de la que se crean dos instancias. En una se crea una pila
de \emph{plugins} para detectar los robots y en la otra una pila para encontrar
la pelota. Cada instancia de \emph{GlobalTracking} solo tomara un nuevo cuadro
del \emph{RingBufer} cuando su pila de plugins haya terminado de procesar el
anterior. El hecho de que solo haya dos instancias de \emph{GlobalTracking}, una
realizando la búsqueda de los robots y la otra la búsqueda de la pelota, genera
una limitación artificial, ya que los cuadros son independientes entre si. Esto
permitiría que se realicen las búsquedas en varios cuadros al mismo tiempo.

El programa consta de cuatro hilos. Uno para la interfaz, uno por cada
\emph{GlobalTracking} y el ultimo para la captura de los cuadros. Esto le
permite un mejor aprovechamiento de los recursos de hardware que
\emph{SSL-Vision}, pero en equipos de mas de cuatro procesadores, estos se ven
desaprovechados.

Bajo las condiciones para las que fue pensado este \emph{framework}, el
desaprovechamiento de los recursos de hardware no es un problema, ya que el
sistema puede procesar un vídeo de 352x228 píxeles a una taza de 60 cuadros por
segundo. Sin embargo, actualmente se utiliza como fuente cuatro cámaras, en
lugar de la cámara única para la cual el sistema fue pensado. Cuando se probó el
\emph{framework} con un vídeo con una resolución de 1080x758 píxeles el número
de cuadros por segundos procesados cayo a 16 cuadros\footnote{Los detalles de
este experimento se detallaran mas adelante}.

