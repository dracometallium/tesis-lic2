% vim: set spell spelllang=es syntax=tex :

\section{Descripción del framework}

Para aumentar el throughput del sistema se aplicaran dos técnicas. En primer
lugar el sistema no estará limitado a tomar solo un cuadro por ves. La segunda
optimización consta en dividir cada cuadro para que cada parte pueda ser
procesada por un thread distinto.

Las clases del framework básico son las siguientes:

\begin{description}

\item[Item]: Esta clase define un tipo genérico de los ítems que serán tratados
	por el sistema.

\item[RingBuffer]: Este es el buffer donde se guardan los ítems generados
	mientras esperan ser procesados. El buffer guarda solo punteros a
	objetos de la clase \emph{Item} y no tiene mecanismos de control que
	permitan acceder la estructura desde múltiples threads al mismo tiempo.
	Cuando se solicita un ítem, se devuelve el puntero a el mas
	recientemente agregado o \textbf{NULL} en caso de que la estructura este
	vacía. Cuando se intenta agregar un nuevo ítem pero la estructura esta
	llena se coloca este en lugar del ítem mas viejo en la estructura y se
	retorna el puntero a este al llamador, delegándole su destrucción. La
	destrucción del ítem se delega al llamador por dos motivos. El primero
	es que el buffer desconoce el tipo real del ítem. La segunda razón es
	que para poder ser utilizado de forma segura, las llamadas a los métodos
	del buffer deben estar dentro de secciones criticas y realizar las
	eliminaciones dentro de estas podría ser muy lento.

\item[Input]: Se trata de una clase que funciona como definición de la interfaz
	de las clases que generan los ítems. Sus métodos principales son
	\emph{run} y \emph{generate}. El método \emph{generate} debe ser re
	implementado por las clases hijas para generar el tipo de ítem
	especifico del sistema. El método \emph{run} es el encargado de generar
	los ítems llamando a \emph{generate} y colocarlos en el
	\emph{RingBuffer}. Este último método puede ser redefinido si la
	aplicación así lo requiere.

\item[ItemSlicer]: Es la clase que define la interfaz de las clases encargadas
	de dividir los ítems. Solo define un método \emph{slice} que recibe como
	parámetro ítem y la cantidad de partes en la que este debe ser dividido.
	Retorna un arreglo de ítems.

\item[Plugin]: Esta clase define una interfaz para los plugins que realizaran
	las distintas partes del procesamiento de la imagen. Solo se define el
	método \emph{process} que tiene como único parámetro un puntero a un
	objeto de la clase \emph{Item}.

\item[PluginStack] Esta es la clase que tomara el ítem y se encarga de
	entregarlo a cada uno de los plugins. Tiene solo dos métodos,
	\emph{addPlugin}, para agregar un plugin, y \emph{process} que tiene
	como parámetro un ítem, para procesar un ítem.

\item[ItemSwitch]: Es la clase encargada de tomar los cuadros del buffer, crear
	dos conjuntos de particiones utilizando la clase descendiente de
	\emph{ItemSlicer} y entregar cada parte a las pilas de plugins. La
	cantidad de cuadros que tomara cada ves y la cantidad de partes en las
	cuales dividirá el ítem se definirán durante su creación.

\end{description}
