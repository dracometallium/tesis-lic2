% vim: set spell spelllang=es syntax=tex :
\ \\
\ \\
\label{pagresum}
\noindent{\LARGE \sc Resumen}\\
\ \\
\ \\

\ \\

\ \\

La \emph{RoboCup} (del inglés \emph{Robot World Cup}) es una competencia donde
dos equipos de robots juegan una versión simplificada del fútbol. Su finalidad
es la de ofrecer un ambiente controlado donde poner a prueba los avances en
distintas áreas de conocimiento como la inteligencia artificial, visión por
computadora y robótica. Existen cinco ligas distintas cuyas características
varían desde la simulación del ambiente y robots, hasta robots humanoides con
visión local, la más antigua de éstas es la liga de tamaño pequeño (también
llamada \emph{SSL} por sus siglas en inglés).

La \emph{SSL} utiliza un sistema de visión global compartido por los dos
equipos.  El sistema procesa cuadros de video y reporta la posición y
orientación de los robots y la posición de la pelota en cada uno de ellos. El
objetivo de este trabajo es proponer un nuevo sistema de visión global por
computadora, alternativo al utilizado actualmente por la \emph{RoboCup} para la
\emph{SSL}, que puede ser aplicado como herramienta educativa para la enseñanza
de visión por computadora y programación paralela sobre máquinas de memoria
compartida basadas en procesadores de propósito general.

Un sistema puede ser considerado para uso educativo en temas de visión por
computadora si el algoritmo que permite la identificación de los objetos de la
escena presenta una clara separación conceptual y fue implementado de forma
simple para permitir su posterior modificación (aún cuando esto implique una
pérdida de rendimiento de la aplicación). Considerando un sistema preexistente
de estas características, se desarrolló uno nuevo capaz de aumentar el
rendimiento mediante el uso eficiente de múltiples unidades de procesamiento y
jerarquía de memoria. El sistema aplica conjuntamente dos estrategias de
paralelización. Una de las estrategias explota el paralelismo dentro de cada
cuadro, dividiendo los cuadros en fragmentos que son procesados de forma
independiente. La otra estrategia se basa en el procesamiento simultáneo de
diferentes cuadros del video.

Se realizó una implementación utilizando el modelo de programación de memoria
compartida \emph{OpenMP} para \emph{C++}. Con el fin de sintonizar la aplicación
para extraer el máximo rendimiento de una determinada plataforma hardware, el
sistema cuenta con diferentes parámetros que permiten modificar el
comportamiento de sus estrategias de paralelización. En un servidor con un
procesador Intel Xeon E5-2630 (6 núcleos y multithreading simultáneo) el sistema
logra una mejora de 5,42$x$ en la cantidad de cuadros por segundo procesados,
con respecto a la ejecución del sistema utilizando un único núcleo (de los 6
disponibles). La posibilidad de modificar el comportamiento de las estrategias
de paralelización es útil para que el estudiante realice experimentación y
analice los resultados buscando explicaciones al impacto en el rendimiento del
sistema.

\vfill
\pagebreak
