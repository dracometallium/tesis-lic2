% vim: set spell spelllang=es syntax=tex :
\ \\
\ \\
\label{pagresum}
\noindent{\LARGE \sc Resumen}\\
\ \\
\ \\

\ \\

\ \\

La \emph{RoboCup} (del inglés \emph{Robot World Cup}) es una competencia donde
equipos de robots juegan una versión simplificada del fútbol. Su finalidad es
la de ofrecer un ambiente controlado donde poner a prueba los avances en
distintas áreas de conocimiento como la inteligencia artificial, visión por
computadora y robótica. Existen cinco ligas distintas cuyas características
varían desde la simulación del ambiente y robots, hasta robots humanoides con
visión local, la más antigua de estas es la liga de tamaño pequeño (también
llamada \emph{SSL} por sus siglas en ingles).

\emph{SSL} utiliza un sistema de visión global compartido por los dos equipos.
El sistema procesa cuadros de vídeo y reporta la posición y orientación de los
robots y la posición de la pelota en cada uno de ellos. En este trabajo se
presenta un nuevo sistema de visión global por computadora alternativo al
utilizado actualmente por la \emph{RoboCup} para la \emph{SSL}, que puede ser
utilizado como herramienta educativa en una asignatura de visión por
Computadora y permite explorar distintas estrategias de paralelización. Se
aplicaron conjuntamente dos estrategias de paralelización. Una de ellas
explota el paralelismo dentro de cada cuadro, dividiendo los cuadros en
fragmentos que son procesados de forma independiente. La otra estrategia se
basa en el procesamiento simultáneo de diferentes cuadros del vídeo.

Se realizó una implementación en OpenMP para C++, basada en plugins para
facilitar su modificación en un ambiente educativo. En un servidor con un
procesador Intel Xeon E5-2630 (6 cores y multithreading simultáneo), la
solución paralela respecto a la solución que utiliza un único núcleo (de los 6
disponibles) logró mejoras máximas de 5,37x en los FPS procesados mientras el
retardo por cuadro se redujo a menos del 20\%.

\vfill
\pagebreak
