% vim: set spell spelllang=es syntax=tex :

\section{Visión por computadora}

La visión por computadora es una disciplina cuyo objetivo principal es definir
un modelo de una escena a partir de imágenes\cite{cvLinda2001}. Como
consecuencia se estudian los métodos captura y procesamiento.

Dependiendo de la aplicación, existen distintas etapas para la obtención de la
información y su procesamiento, pero por lo general se definen los siguientes
pasos\cite{digitalImageProcessing2ed, wikiCV}:

\begin{description}

	\item[Adquisicion de la imagen:] Se trata de la digitalización de la
		imagen a procesar. Esta puede ser obtenida de uno o mas
		sensores. Distintos sensores pueden capturar distintas bandas
		del espectro electromagnético. El resultado puede variar desde
		una imagen 2d monocromática hasta volúmenes tridimensionales.

	\item[Pre procesamiento:] Para que la imagen pueda ser procesada, muchas
		veces es necesario aplicarle transformaciones que no extraen
		información, pero son pre condiciones para las siguientes etapas
		de procesamiento. Algunas de las transformaciones comunes son
		corrección de color, reducción de ruido o cambio de sistema de
		coordenadas de color.

	\item[Extraccion de caracteristicas:] Consiste en el reconocimiento de
		características de bajo nivel como puntos de interés, bordes y
		masas.

	\item[Deteccion y Segmentacion:] Se trata de separar la imagen en sus
		partes constituyentes y seleccionar aquellas de interés.

	\item[Prosesamiento de alto nivel:] En esta etapa se crea finalmente el
		modelo de la escena utilizando los resultados de las etapas
		anteriores.

	\item[Toma de deciciones:] 

\end{description}
