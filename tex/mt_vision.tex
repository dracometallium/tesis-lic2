% vim: set spell spelllang=es syntax=tex :

\section{Visión por computadora}

La disciplina de Visión por Computadora engloba el estudio de métodos de
captura, mejoramiento, y extracción de información a partir de imágenes. Su
objetivo principal es definir un modelo de una escena a partir de
imágenes\cite{cvLinda2001}. En el caso particular del fútbol de robots, el
modelo que se quiere definir describe la posición de los robots y la pelota
dentro de la cancha.

Dependiendo de la aplicación, existen distintas etapas para la obtención de la
información y su procesamiento (Fig. \ref{FALTA3}). En general, se definen los
siguientes pasos\cite{digitalImageProcessing2ed,wikiCV}:

\begin{description}

	\item[Adquisición de la imagen:] Se trata de la digitalización de la
		imagen a procesar. Ésta puede ser obtenida de uno o más
		sensores. Distintos sensores pueden capturar distintas bandas
		del espectro electromagnético. El resultado puede variar desde
		una imagen bidimensional monocromática hasta volúmenes
		tridimensionales.

		En el fútbol de robots de tamaño pequeño, el conjunto de cámaras
		captura una serie de imágenes bidimensionales a color. El
		sistema de coordenadas de la imagen es normalmente \emph{RGB} o
		\emph{YUV}.

	\item[Preprocesamiento:] Para que la imagen pueda ser procesada, muchas
		veces es necesario aplicarle transformaciones que no extraen
		información, pero que son precondiciones para las siguientes
		etapas de procesamiento. Algunas de las transformaciones comunes
		son corrección de color, reducción de ruido o cambio de espacio
		de color.

		Los sistemas de visión global para el fútbol de robots de la
		\emph{SSL} suelen trabajar con el espacio de color \emph{YUV},
		por lo que, de ser necesario, la imagen debe ser convertida a
		este espacio de color. Otro preprocesamiento que puede
		realizarse es la composición de las imágenes del conjunto de
		cámaras en una sola imagen.

	\item[Extracción de características:] Consiste en el reconocimiento de
		características de bajo nivel, como puntos de interés, bordes,
		regiones y masas. Los sistemas para la \emph{SSL} suelen en esta
		etapa reconocer las masas de colores de los parches y la pelota.

	\item[Detección y Segmentación:] Se trata de separar la imagen en sus
		partes constituyentes y seleccionar aquéllas de interés. Los
		sistemas de visión global para fútbol de robots de la \emph{SSL}
		utilizan las masas de color reconocidas en la etapa anterior
		para encontrar los robots y la pelota.

	\item[Procesamiento de alto nivel:] En esta etapa se crea finalmente el
		modelo de la escena utilizando los resultados de las etapas
		anteriores. En el fútbol de robots de la \emph{SSL}, se toman
		las posiciones, dentro de la imagen, de los robots y pelota
		encontrados en la etapa anterior, y se calcula su posición
		dentro de la cancha de acuerdo con la geometría de ésta.

	\item[Toma de decisiones:] Muchas de las aplicaciones de visión por
		computadora, en lugar de sólo entregar información, integran la
		toma de decisiones como parte del procesamiento. En el caso de
		los sistemas de visión global para fútbol de robots, el sistema
		sólo notifica a las computadoras de los equipos la posición de
		los robots y de la pelota.

\end{description}

Nota OSO: Acá re da para afanarse figuritas del trabajo de Guille (pág. 9)

\begin{figure}[!h]

	 \includegraphics[width=\textwidth]{img/FALTA.png}

	\caption{Etapas de procesamiento de un sistema de visión afanado de pág.
	9 del Guille.}

	\label{FALTA3}

 \end{figure}
