% vim: set spell spelllang=es syntax=tex :

\section{Visión por computadora}

La visión por computadora es una disciplina cuyo objetivo principal es definir
un modelo de una escena a partir de imágenes\cite{cvLinda2001}, en el caso del
fútbol de robots, el modelo que se quiere definir es la posición de los robots y
la pelota dentro de la cancha. La disciplina engloba el estudio de métodos
captura, mejoramiento, y extracción de información, un sistema visión por
computadora hace uso de las tres áreas de conocimiento. Dependiendo de la
aplicación, existen distintas etapas para la obtención de la información y su
procesamiento, pero por lo general se definen los siguientes
pasos\cite{digitalImageProcessing2ed,wikiCV}:

\begin{description}

	\item[Adquisición de la imagen:] Se trata de la digitalización de la
		imagen a procesar. Esta puede ser obtenida de uno o mas
		sensores. Distintos sensores pueden capturar distintas bandas
		del espectro electromagnético. El resultado puede variar desde
		una imagen bidimensional monocromática hasta volúmenes
		tridimensionales.
	
		En el fútbol de robots de tamaño pequeño, el conjunto de cámaras
		capturan serie de imágenes bidimensionales a color. El sistemas
		de coordenadas de la imagen es normalmente \emph{RGB} o
		\emph{YUV}.

	\item[Pre procesamiento:] Para que la imagen pueda ser procesada, muchas
		veces es necesario aplicarle transformaciones que no extraen
		información, pero son pre condiciones para las siguientes etapas
		de procesamiento. Algunas de las transformaciones comunes son
		corrección de color, reducción de ruido o cambio de espacio de
		color.

		Los sistemas de visión global para el fútbol de robots de la
		\emph{SSL} suelen trabajar con el espacio de color \emph{YUV},
		por lo que de ser necesario, la imagen debe ser convertida a
		este espacio de color. Otro pre procesamiento que puede
		realizarse es componer las imágenes del conjunto de cámaras en
		una sola imagen.

	\item[Extracción de características:] Consiste en el reconocimiento de
		características de bajo nivel como puntos de interés, bordes,
		regiones y masas. Los sistemas para la \emph{SSL} suelen en esta
		etapa reconocer las masas de colores de los parches y la pelota.

	\item[Detección y Segmentación:] Se trata de separar la imagen en sus
		partes constituyentes y seleccionar aquellas de interés. Los
		sistemas de visión global para fútbol de robots de la \emph{SSL}
		utilizan las masas de color reconocidas en la etapa anterior
		para encontrar los robots y la pelota.

	\item[Procesamiento de alto nivel:] En esta etapa se crea finalmente el
		modelo de la escena utilizando los resultados de las etapas
		anteriores. En fútbol de robots de tamaño pequeño se toman las
		posiciones dentro de la imagen de los robots y pelota
		encontrados en la etapa anterior y se calcula su posición dentro
		de la cancha de acuerdo con la geometría de esta.

	\item[Toma de decisiones:] Muchas de las aplicaciones de visión por
		computadora integran la toma de decisiones como parte del
		proceso, en lugar de solo informar. En el caso de los sistemas
		de visión global para fútbol de robots, el sistema solo notifica
		a las computadoras de los equipos la posición de los robots y
		pelota.

\end{description}
