% vim: set spell spelllang=es syntax=tex :

\section{Metodología}

Esta tesis es de tipo experimental, durante su transcurso se desarrollo un
nuevo sistema de visión global por computadora para fútbol de robots basado en
plugins paralelo capas de aprovechar hardware de distintas cantidades de
núcleos. Para la construcción del nuevo sistema se tomo como base el sistema
descripto en \cite{torres2014}, este es un sistema basado en plugins orientado
al uso educativo. De este se utilizaran los plugins desarrollados modificándolos
para que se integren en el nuevo framework.

Para poder medir el rendimiento del sistema se consideran 2 variables. La
primera es la cantidad de cuadros por segundo (o FPS, de sus siglas en ingles
\emph{Frames per second}) procesados por el sistema. Una taza de cuadros por
segundo más alta se considera beneficiosa, ya que significa que se provee a los
equipos con más información cada segundo. La segunda es el tiempo de espera
máximo de los cuadros. Mientras menor sea tiempo de espera del cuadro, la
información entregada por el sistema sera mas actual, y por lo tanto mas
relevante.
