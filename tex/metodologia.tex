% vim: set spell spelllang=es syntax=tex :

\section{Metodología}

Para la construcción del nuevo sistema de visión global por computadora para
fútbol de robots se tomo como base aquel descripto en \cite{torres2014}.

Esta tesis es de tipo experimental, durante su transcurso se desarrollo un
nuevo sistema de visión global por computadora para fútbol de robots basado en
plugins paralelo capas de aprovechar hardware de distintas cantidades de
núcleos. Para la construcción del nuevo sistema se tomo como base el sistema
descripto en \cite{torres2014}, este es un sistema esta basado en plugins pero
de paralelismo limitado, ya que solo puede hacer uso de 4 núcleos a la vez.

Para comprobar la adaptabilidad del sistema desarrollado por Guillermo Torres en
\cite{torres2014} se grabó un vídeo del que se crearon tres versiones con
resoluciones de 352x228, 800x600 y 1280x720 pixeles, cada uno con una taza de 60
cuadros por segundo. El segundo nos permite verificar el sistema contra los
requerimientos actuales, mientras que el tercero nos permitirá vislumbrar si el
sistema puede adaptarse a una mayor exigencia. Las ejecuciones con cada uno de
estos vídeos se realizaron en una computadora con un procesador Intel Xeon
E5-2630. Este es un procesador de 6 núcleos, cada uno de dos subprocesos
simultáneos a través de la técnica de \emph{simultaneus multithreading} y con un
reloj de 2,30GHz. El equipo cuenta ademas con 16GiB de memoria RAM.

Durante la prueba con el vídeo de resolución 352x228 el sistema pudo procesar
todos los cuadros del vídeo a tiempo (60 cuadros por segundo). Esto era de
esperarse, ya que el sistema fue desarrollado con el objetivo de procesar vídeos
de esa resolución. Sin embargo a pesar de la velocidad y cantidad de los
núcleos, con el vídeo de resolución 800x600 píxeles la cantidad de cuadros por
segundo cayo a 45. Para el vídeo de resolución 1280x720, la reducción del
rendimiento fue mas brusca, decayendo a 14 cuadros por segundo.
