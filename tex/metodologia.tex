% vim: set spell spelllang=es syntax=tex :

\section{Metodología}

La presente tesis es del tipo experimental. Durante su realización se desarrolló
un nuevo sistema de visión global por computadora para fútbol de robots, basado
en plugins, paralelo, capaz de aprovechar hardware multi-core, sin limitaciones
arquitectónicas respecto de la cantidad de núcleos utilizables.

Para la construcción del nuevo sistema se tomó como base el sistema descripto en
\cite{torres2014}. Éste es un sistema basado en plugins, orientado al uso
educativo. Del mismo se utilizaron los plugins desarrollados en dicho trabajo,
modificándolos para que se integren al nuevo framework.

Para poder medir el rendimiento del nuevo sistema se consideraron dos variables.

\begin {itemize}

	\item	La primera variable es la cantidad de cuadros por segundo (o
		FPS, de sus siglas en inglés \emph{Frames per second})
		procesados por el sistema. Una tasa de cuadros por segundo más
		alta se considera beneficiosa, ya que significa que se provee a
		los equipos con más información cada segundo.

	\item	La segunda es el tiempo de espera máximo de los cuadros.
		Mientras menor sea el tiempo de espera del cuadro, la
		información entregada por el sistema será más actual, y por lo
		tanto más relevante.

\end {itemize}

Nota OSO: Acá no va algo de OpenMP? Y la forma de prueba?
