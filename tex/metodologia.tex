% vim: set spell spelllang=es syntax=tex :

\section{Metodología}

\label{metodologia}

La presente tesis es del tipo experimental. Durante su realización se desarrolló
un nuevo sistema de visión global por computadora para fútbol de robots, basado
en plugins, paralelo, capaz de aprovechar hardware multi-core, sin limitaciones
arquitectónicas respecto de la cantidad de núcleos utilizables.

El sistema fue implementado utilizando el modelo de paralelismo de memoria
compartida, utilizando la API de bibliotecas y directivas al compilador
\emph{OpenMP}. Esto permite la creación y control de los hilos de ejecución y
tareas de forma sencilla, y con un código similar al de un programa secuencial.
Se utilizaron los plugins del sistema orientado al uso educativo desarrollado en
\cite{torres2014}, modificándolos para que se integren al nuevo framework.

Para poder medir el rendimiento del nuevo sistema se consideraron dos variables.

\begin {itemize}

	\item	La primera variable es la cantidad de cuadros por segundo (o
		FPS, de sus siglas en inglés \emph{Frames per second})
		procesados por el sistema. Una taza de cuadros por segundo más
		alta se considera beneficiosa, ya que significa que se provee a
		los equipos con más información cada segundo.

	\item	La segunda es el tiempo de espera máximo de los cuadros.
		Mientras menor sea el tiempo de espera del cuadro, la
		información entregada por el sistema será más actual, y por lo
		tanto más relevante.

\end {itemize}
