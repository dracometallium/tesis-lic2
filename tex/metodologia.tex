% vim: set spell spelllang=es syntax=tex :

\section{Metodología}

Como ejemplo para el desarrollo del sistema de visión global se tomo el sistema
desarrollado por Guillermo Torres en \cite{torres2014}. Para comprobar su
adaptavilidad se grabó un vídeo del que se crearon tres versiones con distintas
resoluciones, 1280x720 píxeles, 800x600 píxeles, 352x224 píxeles, cada uno con
una taza de 60 cuadros por segundo. Se realizaron ejecuciones con cada uno de
estos vídeos en una computadora con un procesador Intel Xeon E5-2630. Este es un
procesador de 6 núcleos, cada uno de dos subprocesos simultáneos a través de la
técnica de \emph{simultaneus multithreading} y con un reloj de 2,30GHz. El
equipo cuenta ademas con 16GiB de memoria ram.

Durante la prueba con el vídeo de resolución 352x224 el sistema pudo procesar
todos los cuadros del vídeo a tiempo (60 cuadros por segundo). Esto era de
esperarse, ya que el sistema fue desarrollado con el objetivo de procesar vídeos
de esa resolución. Sin embargo a pesar de la velocidad y cantidad de los
núcleos, con el vídeo de resolución 800x600 píxeles la cantidad de cuadros por
segundo cayo a 45. Para el vídeo de resolución 1280x720, la reducción del
rendimiento fue mas brusca, decayendo a 14 cuadros por segundo.

Para resolver este problema se pueden aplicar tres enfoques distintos:

\begin{itemize}

\item 	Dado que en un vídeo ya decodificado el procesamiento de cada cuadro es
	independiente del procesamiento de los demás, y dada la disponibilidad
	de recursos de computo, se pueden procesar múltiples cuadros al mismo
	tiempo sin que esto genere diferencias en la información obtenida (salvo
	por el orden). Esto permitiría aumentar la cantidad de cuadros por
	segundo obtenidos, aunque el tiempo de procesamiento de cada uno se
	mantenga igual.

\item	Para realizar la búsqueda de los objetos, el cuadro puede ser dividido.
	Esto permitirá realizar la búsqueda en cada zona en paralelo. Para el
	correcto funcionamiento, las zonas deberán tener partes en común que
	dependerán del tamaño de los objetos, ya que no debe suceder que un
	objeto no aparezca completo en ninguna de las zonas. Esto permitiría
	reducir el tiempo de procesamiento de cada cuadro.

\item	Se puede resolver el problema optimizando cada uno de los \emph{plugins}
	del framework, esperando de esta manera que el tiempo de procesamiento
	de cada cuadro se reduzca lo suficiente como para que el sistema pueda
	procesar la mayoría de los cuadros a tiempo.

\end{itemize}

Lamentablemente este último enfoque es muy especifico de la solución para ser
aplicada en el \emph{framework}. Ademas no fomenta la experimentación, ya que
cada \emph{plugin} que se agregue o modifique deberá ser optimizado. El primer y
segundo enfoque son suficientemente generales como para ser aplicados como parte
del \emph{framework}, permite agregar y modificar los \emph{plugins} sin mayores
dificultades, y en el caso en el cual una solución especifica no cumpla con las
condiciones que permitan aplicar alguna de las soluciones se puede limitar el
paralelismo (ya sea no dividiendo el cuadro o procesando solo uno por vez).
