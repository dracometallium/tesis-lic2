% vim: set spell spelllang=es syntax=tex :

\section{Sistemas existentes de visión global para fútbol de robots físicos}


El software de visión global utilizado en la \emph{SSL} de la \emph{Robocup} es
el \emph{SSL-Vision}\cite{sslvision}. Este es un sistema programado en
\emph{C++} basado en plugins, lo que le permite ser extendido fácilmente
mediante estos. Lamentablemente posee algunas desventajas. En primer lugar, el
sistema de captura de cuadros y el de procesamiento están fuertemente
acoplados formando parte del mismo hilo de ejecución. Esto impide hacer uso de
las capacidades de paralelización que ofrece el hardware actual. Como
consecuencia, para poder lograr un desempeño aceptable es necesario que cada
plugin este altamente optimizado, lo que no favorece la experimentación. Por
último, la complejidad del sistema entorpece su uso como herramienta didáctica
para la introducción a la visión por computadora.

En respuesta a estos problemas, en \cite{torres2012, torres2014} se presenta un
nuevo framework, también programado en \emph{C++} y orientado a plugins
destinado al uso educativo y de producción. El framework es general, pero se
provee una implementación de un sistema de visión de fútbol de robots.  La
diferencia principal de este framework es el desacoplamiento del hilo de captura
y los hilos de búsqueda. En el caso del fútbol de robots, el framework tiene dos
hilos de búsqueda (uno para encontrar la pelota y otro para encontrar los
robots), un hilo de captura (desde un vídeo pre grabado o una captura desde una
cámara) y un hilo para la interfaz. Dividir el costo de procesamiento le permite
un mejor aprovechamiento de los recursos de hardware que \emph{SSL-Vision}, pero
en equipos de mas de cuatro procesadores estos se ven desaprovechados.

Bajo las condiciones para las que fue pensado este framework, el
desaprovechamiento de los recursos de hardware no es un problema, ya que el
sistema puede procesar un vídeo de 352x228 píxeles a una taza de 60 cuadros por
segundo. Sin embargo, actualmente se utiliza como fuente cuatro cámaras, en
lugar de la cámara única para la cual el sistema fue pensado. Cuando se probó el
framework con un vídeo con una resolución de 800x600 píxeles el número de
cuadros por segundos procesados cayo a 45 cuadros por segundo.
