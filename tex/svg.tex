% vim: set spell spelllang=es syntax=tex :

\section{Sistemas existentes de visión global para fútbol de robots físicos}

\label{svgExistentes}

El software de visión global utilizado en la \emph{SSL} de la \emph{RoboCup} es
el \emph{SSL-Vision} \cite{sslvision}. Este es un sistema programado en
\emph{C++} basado en plugins, lo que le permite ser extendido fácilmente.
Lamentablemente, posee algunas desventajas con relación a la escalabilidad y al
uso educativo.

\begin{itemize}

	\item 	En primer lugar, el sistema de captura de cuadros y el de
		procesamiento están fuertemente acoplados, formando parte del
		mismo hilo de ejecución. Esto impide hacer uso de este sistema
		con el fin de incorporar la temática de procesamiento paralelo
		en el procesamiento de imágenes.

	\item 	En segundo lugar, para aumentar la eficiencia del sistema, los
		plugins están altamente optimizados, aumentando la complejidad
		del sistema. Esta complejidad dificulta su uso como herramienta
		didáctica para la introducción al tema de la visión por
		computadora.

\end{itemize}

Abordando estos problemas, en \cite{torres2014} se presenta un sistema destinado
al uso educativo, basado en pilas de plugins, y también programado en
\emph{C++}. El sistema está planteado como un framework de construcción de
aplicaciones que permite elaborar diferentes estrategias y medir su rendimiento,
sin necesidad de construir una aplicación desde cero para cada una de ellas. En
el trabajo citado se ha validado el framework mediante la construcción y prueba
de una instancia de aplicación de servidor de vídeo para un campeonato de fútbol
de robots reales.

La capacidad de respuesta a las restricciones de tiempo real del servidor así
construido fueron probadas en condiciones acordes a las presentadas en las
competencias. Se estableció un tiempo máximo de procesamiento de cuadro
admisible. Se midió el porcentaje de cuadros para los cuales el tiempo de
procesamiento superaba este máximo, hallando que este porcentaje quedaba dentro
de los objetivos de tolerancia.

La programación del framework propone un enfoque multi-hilo con el propósito de
mejorar la escalabilidad en microprocesadores multi-core. Utiliza dos hilos de
búsqueda de objetos: uno para encontrar la pelota y otro para encontrar los
robots; un hilo de captura de cuadros de vídeo (desde un vídeo pre grabado o una
captura desde una cámara), y finalmente uno para la interfaz de usuario. Su
grado de paralelismo es mejor que el de \emph{SSL-Vision}; pero aún así, al
organizar el cómputo en tan sólo cuatro hilos en total, su diseño está acoplado
a una arquitectura de, a lo sumo, cuatro núcleos. Por lo tanto, no resulta capaz
de aprovechar plenamente las capacidades de plataformas de hardware con mayores
recursos.

Bajo las condiciones para las que fue diseñado este framework, el
desaprovechamiento de los recursos de hardware no es un problema, ya que, sobre
el hardware utilizado para las pruebas, el sistema puede procesar un vídeo de
352$\times$228 píxeles a una tasa de 60 cuadros por segundo. Sin embargo, en las
competencias actuales se utilizan cuatro cámaras como origen de los datos de
vídeo, en lugar de la cámara única para la cual el sistema fue pensado.

Como punto de partida de la presente tesis, nos formulamos la pregunta de si el
framework soportaría este nuevo caudal de información. En este caso, si cada
cámara captura un vídeo con una resolución de 325$\times$228, el vídeo compuesto
resultante tendrá una resolución levemente inferior a 800$\times$600 píxeles.
Al probar el servidor de vídeo, construido sobre el framework, con un vídeo de
esa resolución el número de cuadros por segundo procesados cayó a 45 cuadros por
segundo.

Esta diferencia en prestaciones muestra que la utilidad del sistema se ve
comprometida a medida que el volumen de los datos aumenta. Como las cuatro
cámaras de las competencias actuales pueden ofrecer una carga agregada todavía
superior a la utilizada en el experimento, resulta de interés abordar una mejora
de escalabilidad en la arquitectura de software del framework, a fin de
permitirle hacer frente a los cambios de escala en el problema.
