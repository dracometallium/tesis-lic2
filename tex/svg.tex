% vim: set spell spelllang=es syntax=tex :

\section{Sistemas existentes de visión global para fútbol de robots físicos}


El software de visión global utilizado en la \emph{SSL} de la \emph{Robocup} es
el \emph{SSL-Vision}\cite{sslvision}. Este es un sistema programado en
\emph{C++} basado en plugins, lo que le permite ser extendido fácilmente
mediante estos. Lamentablemente, posee algunas desventajas en relación a la
escalabilidad y al uso educativo. En primer lugar, el sistema de captura de
cuadros y el de procesamiento están fuertemente acoplados formando parte del
mismo hilo de ejecución. Esto impide hacer uso de las capacidades de
paralelización que ofrece el hardware actual, limitando su desempeño y
escalabilidad. Para aumentar la eficiencia del sistema los plugins están
altamente optimizados aumentando la complejidad del sistema. Esta complejidad
entorpece su uso como herramienta didáctica para la introducción a la visión por
computadora.

En \cite{torres2014} se presenta un sistema destinado al uso educativo, basado
en pilas de plugins, y también programado en \emph{C++}. El framework tiene dos
hilos de búsqueda de objetos, uno para encontrar la pelota y otro para encontrar
los robots, un hilo de captura de cuadros de vídeo (desde un vídeo pre grabado o
una captura desde una cámara), y un hilo para la interfaz de usuario. Su grado
de paralelismo es mejor que el de \emph{SSL-Vision} pero aún así no permite
escalar a más de cuatro núcleos.

Bajo las condiciones para las que fue pensado este framework, el
desaprovechamiento de los recursos de hardware no es un problema, ya que el
sistema puede procesar un vídeo de 352x228 píxeles a una taza de 60 cuadros por
segundo. Sin embargo, actualmente se utiliza como fuente cuatro cámaras, en
lugar de la cámara única para la cual el sistema fue pensado. Cuando se probó el
framework con un vídeo con una resolución de 800x600 píxeles el número de
cuadros por segundos procesados cayo a 45 cuadros por segundo.
