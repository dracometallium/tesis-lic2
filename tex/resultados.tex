% vim: set spell spelllang=es syntax=tex :

\section{Metodología experimental}

Como se menciono con anterioridad, para comprobar el funcionamiento del nuevo
framework se procesaron los vídeos de resolución 352x228, 800x600 y 1280x720, en
una computadora con un procesador Intel Xeon E5-2630. Este es un procesador de 6
núcleos, cada uno de dos subprocesos simultáneos a través de la técnica de
\emph{simultaneus multithreading} y con un reloj de 2,30GHz. El equipo cuenta
ademas con 16GiB de memoria RAM. Los resultados del vídeo de resolución 352x228
píxeles no serán tratados mas adelante debido a que la carga al sistema es
reducida y no aporta información que no aporten los otros dos vídeos.

Se realizaron pruebas con cada vídeo, variando la cantidad de partes en las que
fueron divididos los cuadros de 1 a 24, y la cantidad de hilos de búsqueda entre
1 y 11. De la primera variable se tomaron todos los valores enteros dentro del
rango ya que el área del fragmento varia de forma irregular entre valores
adyacentes. De la segunda variable solo se tomaron los valores impares, ya que
el cambio de esta es mas gradual, haciendo innecesario tomar todo el conjunto
para analizar las tendencias.

En cada test se ejecuto el programa durante 13 segundos. Durante los últimos 10
segundos se contó la cantidad de cuadros procesados, y de cada uno de estos se
registro el retardo entre su creación su procesamiento. Los primeros 3 segundos
se ignoraron con el fin de permitir que la ejecución se estabilizara. Cada test
se realizo 10 veces.

Cada prueba se ejecuto dos veces, la primera buscando el mínimo de la cantidad
máxima de cuadros por segundo que soporta el sistema bajo cada configuración. En
la segunda se limito la cantidad de cuadros por segundo de cada configuración,
según lo encontrado en la prueba anterior, y se registro el tiempo re
procesamiento máximo para cada de estas.

Durante el desarrollo de la aplicación se probaron tres distintas
implementaciones. En la primera implementación, el framework ejecutaba las
distintas pilas de plugins en tareas separadas. En la segunda implementación el
framework fue modificado para que una sola tarea ejecutara todas las pilas de
plugins sobre un mismo fragmento. Para la tercera implementación se trabajo
sobre el mismo framework que la segunda, pero se unieron las pilas de búsquedas
de robots y la de búsqueda de pelota. De estas la última fue la que produjo
resultados mas satisfactorios, por lo que sera sobre los resultados de esta que
trabajaremos a continuación.

\section{Resultados}

//tablas de tiempo y retardo.

//Poner gráfico de retardo/cuadros
