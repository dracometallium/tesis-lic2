% vim: set spell spelllang=es syntax=tex :

\section{Metodología experimental}

Como se menciono con anterioridad, para comprobar el funcionamiento del nuevo
framework se procesaron los vídeos de resolución 640x480, 800x600 y 1280x720, en
una computadora con un procesador Intel Xeon E5-2630. Este es un procesador de
6 núcleos, cada uno de dos subprocesos simultáneos a través de la técnica de
\emph{simultaneus multithreading} y con un reloj de 2,30GHz. El equipo cuenta
ademas con 16GiB de memoria ram.

Se realizaron pruebas con cada vídeo, variando la cantidad de partes en las que
fueron divididos los cuadros y la cantidad de procesadores disponibles a la
aplicación. Para cada caso, se estableció la cantidad máxima de tareas en
ejecución simultanea como la cantidad de procesadores menos uno. El procesador
no usado por las tareas de búsqueda se dejo para ser compartido entre el hilo
de captura y el hilo que crea las tareas de búsqueda.

En cada test se ejecuto el programa durante 15 segundos. Durante los
últimos 10 segundo se contó la cantidad de cuadros procesados, y de cada uno de
estos se registro el retardo entre su creación su procesamiento. Los primeros 5
segundos se ignoraron con el fin de permitir que la ejecución se estabilizara.
Cada test se realizo 10 veces. Para eliminar valores atípicos se elimino el
mínimo y el máximo, y sobre los restantes se calculo el promedió.

\section{Resultados}

//tablas de tiempo y retardo.

//Poner gráfico de retardo/cuadros
