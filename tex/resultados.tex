% vim: set spell spelllang=es syntax=tex :

\section{Metodología experimental}

Como se menciono con anterioridad, para comprobar el funcionamiento del nuevo
framework se procesaron los vídeos de resolución 352x228, 800x600 y 1280x720, en
una computadora con un procesador Intel Xeon E5-2630. Este es un procesador de 6
núcleos, cada uno de dos subprocesos simultáneos a través de la técnica de
\emph{simultaneus multithreading} y con un reloj de 2,30GHz. El equipo cuenta
ademas con 16GiB de memoria RAM.

Se realizaron pruebas con cada vídeo, variando la cantidad de partes en las que
fueron divididos los cuadros y la cantidad de procesadores disponibles a la
aplicación. Para cada caso, se estableció la cantidad máxima de tareas en
ejecución simultanea como la cantidad de procesadores menos uno. El procesador
no usado por las tareas de búsqueda se dejo para ser compartido entre el hilo de
captura y el hilo que crea las tareas de búsqueda.

En cada test se ejecuto el programa durante 15 segundos. Durante los últimos 10
segundo se contó la cantidad de cuadros procesados, y de cada uno de estos se
registro el retardo entre su creación su procesamiento. Los primeros 5 segundos
se ignoraron con el fin de permitir que la ejecución se estabilizara. Cada test
se realizo 10 veces. Para eliminar valores atípicos se elimino el mínimo y el
máximo, y sobre los restantes se calculo el promedió.

Durante el desarrollo de la aplicación se probaron tres distintas
implementaciones. En la primera implementación, el framework ejecutaba las
distintas pilas de plugins en tareas separadas. En la segunda implementación el
framework fue modificado para que una sola tarea ejecutara todas las pilas de
plugins sobre un mismo fragmento. Para la tercera implementación se trabajo
sobre el mismo framework que la segunda, pero se unieron las pilas de búsquedas
de robots y la de búsqueda de pelota.

\section{Resultados}

//tablas de tiempo y retardo.

//Poner gráfico de retardo/cuadros
