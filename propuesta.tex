% vim: set spell:
\documentclass[a4paper]{article}

\usepackage{babelbib}

\usepackage{url}

\usepackage[spanish]{babel}

\usepackage[utf8]{inputenc}

\title{Propuesta de tesis:\\ Implementación paralela de un sistema de visión global
por computadora}

%\author{Rodrigo S. Cañibano \emph{(rcanibano@fi.uncoma.edu.ar)}}

\author{}

\date{}

\begin{document}

\maketitle

\section{Datos generales}

\begin{itemize}

	\item{Tema del proyecto: Implementación paralela de un sistema de visión
		global por computadora}

	\item{Tipo de tesis: Experimental}

	\item{Nombre del alumno: Rodrigo S. Cañibano}

	\item{Director de tesis: Eduardo Grosclaude}

	\item{Codirector de tesis: Javier Balladini}

	\item{Fecha de presentación de la propuesta: 2 Febrero 2015}

\end{itemize}

\section{Motivación y marco de trabajo}

El fútbol de robots de la liga de tamaño pequeño (\emph{SSL}, del ingles
\emph{Small Size League})\cite{sslrules2014} es una competencia en la que se
enfrentan dos equipos de robots controlados cada uno por una computadora, que
perciben el ambiente a través de un sistema de visión global centralizado. Este
sistema de visión consta de un conjunto de cámaras montadas sobre distintas
áreas del campo de juego conectadas a una computadora donde se ejecuta el
sistema de visión. El sistema detecta la posición y orientación de cada uno de
los robots, y la posición de la pelota, y reporta esta información a las
computadoras que controlan los equipos. El uso de este sistema centralizado
permite a los participantes abstraerse de los problemas de la visión por
computadora y enfocarse en la estrategia del juego. Además, permite que las
tareas de calibración y montaje de las cámaras se realicen una sola vez para
cada campo de juego en vez de para cada partido.

Consideramos que el juego de fútbol de robots puede ser utilizado 
como marco para la enseñanza y aprendizaje del área de conocimiento de visión por 
computadoras en instituciones educativas. Actualmente, las competencias de 
la \emph{SSL} utilizan el sistema de visión
global SSL-Vision\cite{sslvision}, un framework basado en plugins
desarrollado en C++. Sin embargo, su uso es complejo para ser utilizado 
como herramienta educativa. En respuesta a esta problemática, en
\cite{torres2014}, Guillermo Torres propone un nuevo framework, también basado en plugins
y desarrollado en C++, destinado al uso educativo y de producción. El trabajo aquí
propuesto consiste en ofrecer mejoras de rendimiento a este sistema.

\section{Antecedentes y objetivo}

Un sistema de visión global (SVG) reporta a cada equipo la
posición y orientación de cada uno de los jugadores, y la posición de la pelota,
y para ser de utilidad se requiere que los reportes sean precisos, rápidos y
frecuentes.

Originalmente el tamaño de la cancha era de 4,9$m$x3,4$m$, lo que permitía que
todo el campo de juego fuera observado con una sola cámara. Actualmente, existen
dos tipos de canchas en los partidos de la \emph{SSL}: las canchas de tamaño
simple, con un tamaño de 6,05$m$x4,05$m$ para las cuales se utilizan dos
cámaras, una sobre cada media cancha, y las de tamaño doble, con un tamaño de
8,09$m$x6,05$m$, que utilizan cuatro cámaras, una por cada mitad de cada media
cancha. Se espera que para el 2015 las canchas de tamaño doble sean las
utilizadas de forma predeterminada. Con este cambio se espera permitir la
exploración de nuevas tácticas por parte de los equipos, ya que mayor área
permitirá mayor movilidad. Sin embargo, esto trae aparejado el problema de que
en las canchas de tamaño doble se deben procesar cuatro veces la información que
en las canchas originales, con una resolución total levemente inferior a 800x600
píxeles.

De esta forma definimos el objetivo (general) de este trabajo que es la
propuesta de un SVG paralelo, destinado al uso educativo y de producción, que
soporte el procesamiento de una entrada de vídeo con una resolución de 800x600
píxeles y 28 cuadros por segundo. La propuesta involucra tanto el diseño como la
implementación del sistema de visión.



\section{Metodología, objetivos específicos y plan de actividades}

La metodología consta en el desarrollo de un SVG paralelo basado en el SVG propuesto en
\cite{torres2014}, buscando mejorar su rendimiento mientras se intenta preservar
sus características educativas.

Para llevar a cabo la propuesta se deberán lograr los siguientes objetivos
específicos:

\begin{enumerate}

	\item{Identificar las oportunidades e inhibidores de paralelismo en el
		sistema de \cite{torres2014}.}

	\item{Encontrar una estrategia de paralelización que pueda explotar las
		oportunidades de paralelización encontradas en el punto 1.} 

	\item{Realizar la implementación del sistema de visión paralelo.}

	\item{Contrastar el comportamiento del sistema desarrollado con los
		requerimientos definidos en el objetivo general.}

\end{enumerate}

Estos objetivos intentaran ser alcanzados mediantes las siguientes actividades:

\begin{enumerate}

	\item{Leer la tesis y otras publicaciones relacionadas al sistema
		desarrollado en \cite{torres2014} (objetivo 1).}
		
	\item{Rever conceptos de la computación paralela (objetivo 1).}

	\item{Diseñar una solución paralela del SVG (objetivo 2).}

	\item{Seleccionar el o los modelos de programación paralela adecuados
		para implementar la solución (objetivo 3).}

	\item{Escritura del código paralelo de la aplicación (objetivo 3).}

	\item{Realizar experimentaciones para obtener datos de rendimiento
		(objetivo 4).}

	\item{Analizar los resultados (objetivo 4).}

	\item{Escribir la tesis}

\end{enumerate}

El cronograma pretendido es el siguiente:


\begin{tabular}{ | c |c |c |c |c |c |c | }
	
	\hline \hline

	& Mes 1 & Mes 2 & Mes 3 & Mes 4 & Mes 5 & Mes 6 \\
	
	\hline
	
	Actividad 1 & X & X & & & & \\
	
	\hline
	
	Actividad 2 & X & X & & & & \\

	\hline
	
	Actividad 3 & & X & & & & \\

	\hline
	
	Actividad 4 & & X & & & & \\

	\hline
	
	Actividad 5 & & & X & X & & \\

	\hline
	
	Actividad 6 & & & & & X & \\

	\hline
	
	Actividad 7 & & & & & X & \\

	\hline
	
	Actividad 8 & & X & X & X & X & X\\
	
	\hline \hline

\end{tabular}


\bibliographystyle{babplain}

\bibliography{biblio}

\end{document}
